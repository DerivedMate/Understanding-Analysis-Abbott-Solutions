%% 6.6 %%

\section{Taylor Series}
\setcounter{exercise}{0}

\bx{
If we know that $\arctan(x)$ is continuous,
we just need to show the series is also continuous over the interval $\pbra{-1, 1}$.
We know that the series converges at 1, which by Abel's Theorem gives us 
uniform convergence over $\pbra{-1, 1}$. Every $a_nx^n$ is also continuous,
which means that the series is continuous over this interval as well.
Since the series and $\arctan(x)$ have the same value over $[0, 1)$,
we can conclude they must have the same value at $1$ as well.

We get the property that 
\begin{equation*}
  \arctan (1) = \frac{\pi}{4} = 
  1 - \frac{1}{3} + \frac{1}{5} - \frac{1}{7} + \cdots
  = \sum_{n=1}^\infty (-1)^{n+1}\frac{1}{2n-1}
\end{equation*}
}

\bx{
We know 
\begin{equation*}
  \dv{x} \ln(1+x) = \frac{1}{1+x} = 1 -x + x^2 - x^3 + \cdots
\end{equation*}
So if we take the integral,
\begin{align*}
  \ln(1+x) 
  &= \int_{0}^x \pa{1 -x + x^2 - x^3 + \cdots } \dd{x}\\
  &= 
  x - \frac{x^2}{2} + \frac{x^3}{3} - \frac{x^4}{4} + \cdots\\
  &=
  \sum_{n=1}^\infty (-1)^{n+1}\frac{x^n}{n}
\end{align*}
This is value for $\abs{x} < 1$, if we want to use the geometric series identity,
and also converges for $x=1$ since it is the alternating harmonic series, so 
converges over $x \in (-1, 1]$.
}

\bx{
\ea{
\item Taking derivatives
\begin{equation*}
  f'(x) = a_1 + 2a_2x + 3a_3x^2 + 4a_4x^3 + \cdots
\end{equation*}
so we see $f'(x) = a_1$.

\item We can show that after $f^{(n)}(x)$,
the coefficient of $x^n$ is 
\begin{equation*}
  \dv[n]{x} (a_nx^n) = n! \cdot a_n,
\end{equation*}
by the power rule of differentiation.

Then we have 
\begin{equation*}
  f^{(n)}(x) = n! \cdot a_n + x\pa{\text{stuff}} \Rightarrow f^n(0) = n! \cdot a_n
\end{equation*}
so therefore \begin{equation*}
  a_n = \frac{f^{(n)}(0)}{n!}
\end{equation*}
We can use Theorem 6.3.3 to justify our actions,
since we know all the terms in the series are differentiable, 
since they are in the form $a_nx^n$, and also we know the point $x_0 = 0$
has a convergent point, so therefore $\pa{f^{(n)}}' = \sum (b_n)'$ where $b_n$
are the terms of $f^{(n)}$.
}
}

\bx{
We can try a few terms,
\begin{align*}
  a_0 &= \sin(0) = 0\\
  a_1 &= \frac{\cos(0)}{1!} = 1\\
  a_2 &= \frac{-\sin(0)}{2!} = 0\\
  a_3 &= \frac{-\cos(0)}{3!} = -\frac{1}{3!}\\
  a_4 &= \frac{\sin(0)}{4!} = 0\\
  a_5 &= \frac{\cos(0)}{5!} = \frac{1}{5!}\\
  &\cdots
\end{align*}
and we can verify the pattern for the Taylor series for $\sin(x)$.
}

\bx{
We know by Lagrange's remainder theorem that $\exists c$ such that
\begin{equation*}
  E_N(x) = \sin(x) - S_N(x) \leq \abs{
    \frac{
      \cos(c)
    }{
      (N+1)!
    }
  }2^{N+1} \leq 
  \frac{2^{N+1}}{(N+1)!}
\end{equation*}
and this quantity $\to 0$ as $N \to \infty$ since factorial
grows faster than exponential. Therefore, $E_N(x) \to 0$, uniformly.
Our argument works for any interval $[-R, R]$.
}

\bx{
\ea{
\item We can find 
\begin{align*}
  a_0 &= e^0 = 1                      \\
  a_1 &= \frac{e^0}{1!} = 1           \\
  a_2 &= \frac{e^0}{2!} = \frac{1}{2!}\\
  a_3 &= \frac{e^0}{3!} = \frac{1}{3!}\\
  a_4 &= \frac{e^0}{4!} = \frac{1}{4!}\\
  a_5 &= \frac{e^0}{5!} = \frac{1}{5!}\\
  &\cdots
\end{align*}
So therefore 
\begin{equation*}
  e^x = \sum_{n=0}^\infty \frac{x^n}{n!}.
\end{equation*}
On any interval $[-R, R]$, we can bound $e^c \leq M$ for $c \in [-R, R]$.
Now, 

\begin{equation*}
  \abs{E_N(x)} 
  = 
  \abs{
    \frac{e^c}{(N+1)!}x^{N+1}
  } \leq \frac{M\cdot R^{N+1}}{(N+1)!}
\end{equation*}
this shows $\abs{E_N(x)} \to 0$, so it converges uniformly to 0.

\item We can verify that 
\begin{align*}
  (e^x)'
  &= (1)' + \sum_{n=1}^\infty \frac{nx^{n-1}}{n!}\\
  &= 0 + \sum_{n=1}^\infty \frac{x^{n-1}}{(n-1)!}\\
  &= \sum_{n'=0}^\infty \frac{x^{n'}}{(n')!}\tag{$n' = n-1$}
\end{align*}

\item We can make 
\begin{equation*}
  e^{-x} = \sum_{n=0}^\infty (-1)^n\frac{x^n}{n!}.
\end{equation*}
and find that 
\begin{align*}
  e^x \cdot e^{-x}
  &= 1 + 
  x\pa{
    1 - 1
  } + x^2\pa{
    2\frac{1}{2!} - \pa{\frac{1}{1!}}^2
  } + x^3\pa{
    \frac{1}{3!}(1-1) + \frac{1}{2!}(1-1)
  } + x^4\pa{
    \frac{2}{4!} - \frac{2}{3!} + \pa{\frac{1}{2!}}^2
  } + \cdots \\
  &= 1 + 0 + 0 + \cdots\\ 
  &= 1,
\end{align*}
which is expected since $e^xe^{-x} = e^0 = 1$.

For a more rigorous justification, it is not hard to show for 
$x^k$ $k$ odd, all the terms will cancel each other out.
For even powers, the task is more difficult.

The easiest way to show they cancel out is to notice that
the coefficients of $x^n$ after the multiplication is just 
\begin{equation*}
  \frac{1}{n!} (-1 + 1)^n = 0
\end{equation*}
}
}

\bx{
Not sure if this problem has a typo...but I think the author meant 
\begin{equation*}
  E_N(x) = f(x) - S_N(x)
\end{equation*}
and not $f_N(x)$...whatever that is supposed to mean.
\begin{align*}
  E^{(n)}_N(0) 
  &= f^{(n)}(0) - S^{(n)}_N(0)\\
  &= f^{(n)}(0) - a_N \cdot N!\\
  &= f^{(n)}(0) - \frac{f^{(n)}(0)}{N!} \cdot N!\\
  &= 0
\end{align*}
}

\bx{
We have 
\begin{align*}
  E_N(x) &= \frac{x^{N+1}}{N+1} \frac{E'_N(x_1)}{x^N_1}\\
  &= \frac{x^{N+1}}{N+1} \cdot \frac{1}{N}\cdot\frac{E^{(2)}_N(x_2)}{x^{N-1}_2}\\
  &= \frac{x^{N+1}}{N+1} \cdot \frac{1}{N}\cdot\frac{1}{N-1}\cdot
  \frac{E^{(3)}_N(x_3)}{x^{N-2}_3}\\
  &= \frac{x^{N+1}}{(N+1)!}E^{(N+1)}(x_{N+1})\\
  &= \frac{x^{N+1}}{(N+1)!}\pbra{
    f^{(N+1)}(x_{N+1}) - S^{(N+1)}_N(x_{N+1})
  }\\
  &= \frac{
    f^{(N+1)}(x_{N+1})
  }{
    (N+1)!
  }x^{N+1} 
\end{align*}
For the last step, $S_N$ highest degree term is $x^N$,
so after $N+1$ derivatives, $S^{(N+1)}_N(x) = 0$.
We see the overall strategy is just repeatedly apply the 
GMVT between $x=x_k, 0$, and we can find 
\begin{equation*}
  x_2 \in (0, x_1), x_3 \in (0, x_2), \dots, x_k \in (0, x_{k-1}),
\end{equation*}
So we set $c = x_{N+1}$ to be used in the theorem.

The negative direction is basically the same proof, just we choose 
$x_k \in (x_{k-1}, 0)$.
}

\bx{
Computing
\begin{align*}
  g'(0) 
  &= \lim_{x \to 0} \frac{
    e^{-1/x^2} - 0
  }{
    x
  }\\
  &= \lim_{x \to 0} \frac{
    x^{-1}
  }{
    e^{1/x^2} 
  }\\
  &= \lim_{x \to 0} \frac{
    -x^{-2}
  }{
    e^{-1/x^2}(2x^{-3})
  } \tag{L'Hopital's}\\
  &= \lim_{x\to 0} \frac{
    x
  }{
    2e^{1/x^2}
  }\\
  &= 0
\end{align*}
}

\bx{
For $x \neq 0$,
\begin{align*}
  g'(x) &= 
    \frac{2x^{-3}}{e^{1/x^2}}\\
  g''(x) &= 
    \frac{4x^{-6} - 6x^{-4}}{e^{1/x^2}}\\
  g'''(x) &= 
    \frac{8x^{-9} - 36x^{-7} + 24x^{-5}}{e^{1/x^2}}
\end{align*}
We can use the quotient rule to show that for 
\begin{equation*}
  \dv{x} \frac{f_k}{e^{1/x^2}} = 
  \frac{
    f'_k e^{1/x^2} - e^{1/x^2}(-2x^{-3})f_k
  }{
    \pa{e^{1/x^2}}^2
  } = 
  \frac{
    f'_k + 2x^{-3}f_k
  }{e^{1/x^2}}
\end{equation*}
So we can make a sequence-like definition for $g'(x)$.
}

\bx{
Computing
\begin{align*}
  \lim_{x\to 0} \frac{
    \frac{2x^{-3}}{e^{1/x^2}} - 0
  }{
    x - 0
  }
  &=
  \lim_{x\to 0} \frac{
    2x^{-4}
  }{
    e^{-1/x^2}
  }\\
  &=
  \lim_{x\to 0} \frac{
    -8x^{-5}
  }{
    e^{1/x^2}(-2x^{-3})
  }\\
  &=
  \lim_{x\to 0} \frac{
    4x^{-2}
  }{
    e^{1/x^2}
  }\\
  &=
  \lim_{x\to 0} \frac{
    -8x^{-3}
  }{
    e^{1/x^2}(-2x^{-3})
  }\\
  &=
  \lim_{x\to 0} \frac{
    4
  }{
    e^{1/x^2}
  }\\
  &= 0
\end{align*}
We can see in general, the L'Hopital's will produce enough $-2x^{-3}$
to cancel out any factors in the numerator, until the numerator 
has a non-negative degree polynomial in $x$, which means 
\begin{equation*}
  \lim_{x\to 0} \frac{
    f(x)
  }{
    e^{1/x^2}   
  }
  = 0/\infty = 0 = g^{(n)}(0)
\end{equation*}
}

\bx{
$g$ is infinitely differentiable, since we found a general formula for its
$n^\text{th}$ derivative.

We found that $g^{(k)}(0) = 0$, which means the Taylor Series will be 
$a_n = 0$. Of course, since this is $f(x) = 0$, this converges everywhere.
However, we clearly see that $g(x) \neq 0$, so the Taylor series does not work
for this example.
}