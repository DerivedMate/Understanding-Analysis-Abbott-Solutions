%% 6.3 %%

\section{Uniform Convergence and Differentiation}
\setcounter{exercise}{0}

\bx{
\ea{
\item \begin{equation*}
  \abs{
    \frac{
      \sin(nx)
    }{
      n
    }
  } \leq \frac{1}{n} \tag{$\abs{\sin(y)} < 1$}
\end{equation*}
We first find
\begin{equation*}
  h'_n(x) = \frac{
    n\cos(nx)
  }{
    n
  } = \cos(nx)
\end{equation*}
this sequence has issues converging when $n\to \infty$ for $\cos(nx)$,
but does converge when $x = 2\pi k$, since $\cos(2\pi k n) = 1$.
The official solutions says $x = \pi/2 + \pi k$, which works 
but I think misses on the $x = 2\pi k$ solution.
We can verify $x = \pi/2$ for example, 
\begin{equation*}
  \cos(\pi/2), \cos(3\pi/2), \cos(5\pi/2), \cos(7\pi/2), \dots \Rightarrow 
  0, 0, 0, 0, \dots
\end{equation*}

\item Use $h_n(x) = \frac{\sin(n^2x)}{n}$, then 
\begin{equation*}
  h'_n(x) = n\cos(n^2x)
\end{equation*}
}
}

\bx{
\ea{
\item We see that $\lim g_n = 0$ since $x^n \leq 1$, while the 
denominator grows with $n$ linearly. Then,
\begin{equation*}
  \abs{
    \frac{
      x^n
    }{
      n
    } - 0
  } \leq \frac{1}{n},
\end{equation*}
which shows uniform convergence.

$g(x) = 0$, so $g'(x) = 0$.

\item 
We have 
\begin{equation*}
  g'_n(x) = \frac{
    nx^{n-1} 
  }{
    n
  } = x^{n-1}
\end{equation*}
which exists $\forall x \in [0, 1]$.

We can show that $g'_n(x)$ converges to $0$ for $x < 1$, and $1$ if $x = 1$,
so we 
run into problems for convergence if $x = 1$.

We have \begin{equation*}
  h = \lim g'_n = \begin{cases}
    0, &x \in [0, 1)\\
    1, &x = 1
  \end{cases}
\end{equation*}
which is $\neq g'(x) = 0$.

This tells us in general, 
\begin{equation*}
  \lim g'_n = h \neq g'
\end{equation*}
}
}

\bx{
We have 
\begin{equation*}
  f = \lim f_n = 0
\end{equation*}
and
\begin{equation*}
  f'_n(x) = \frac{
    1+nx^2 - x(2nx)
  }{
    \pa{1+nx^2}^3
  } = \frac{
    1-nx^2
  }{
    \pa{1 + nx^2}^2
  }
\end{equation*}
% Using a similar strategy from Exercise
% \ref{chap6:fnx_optima},
% we take the derivative of $f'_n(x)$,
% \begin{align*}
%   f''_n(x)
%   &= \frac{
%     \pa{-2nx}\pa{
%       1+nx^2
%     }^2 - 
%     \pa{
%       1-nx^2
%     }\pa{
%       2\pa{
%         1+nx^2
%       }\pa{
%         2nx
%       }
%     }
%   }{
%     \pa{1+nx^2}^4
%   }\\
%   &= \frac{
%     \pa{-2nx}\pa{
%       1+nx^2
%     }^2 - 
%     \pa{
%       4nx
%     }
%     \pa{
%       1-nx^2
%     }\pa{
%       1+nx^2
%     }
%   }{
%     \pa{1+nx^2}^4
%   }\\
%   \Rightarrow
%   0 &= \pa{-2nx}\pa{
%     1+nx^2
%   }^2 - 
%   \pa{
%     4nx
%   }
%   \pa{
%     1-nx^2
%   }\pa{
%     1+nx^2
%   }\tag{Optima when this $=0$}\\
%   &= x\pbra{
%     \pa{1+nx^2} + 2\pa{
%       1-nx^2
%     }
%   }\\ 
%   &= x \pbra{
%     3 - nx^2
%   }
% \end{align*}
% So we see the optima are at 
% \begin{equation*}
%   x = 0, \pm \sqrt{\frac{3}{n}},
% \end{equation*}
% and at these values, $f'_n(x) = 1, -\frac{1}{8}$,
% so we know which one corresponds to the max and min. 
% We also just have to show that for $x \to \pm \infty$, that this function
% is bounded to show that we can bound $f'_n(x)$.

We know $f(x) = 0$, so $f'(x) = 0$.
We see that for $x = 0$, 
\begin{equation*}
  f'_n(0) = \frac{
    1-0
  }{
    \pa{1+0}^2
  } = 1 \neq 0 = f'(x),
\end{equation*}
but for all other values of $x$, we end up with 
\begin{equation*}
  \frac{\text{stuff}}{
    \text{less dominant terms} + n^2x^4
  } \to 0 = f'(x)
\end{equation*}
}

\bx{
\ea{
\item If we algebraically calculate the limit,
\begin{equation*}
  g(x) = \lim_{n\to\infty} \frac{nx+x^2}{2n} = \frac{x}{2}
\end{equation*}
Then 
\begin{equation*}
  g'(x) = \frac{1}{2}
\end{equation*}

\item We can find that 
\begin{equation*}
  g'_n(x) = \frac{
    n+2x
  }{2n} = \frac{1}{2} + \frac{x}{n}
\end{equation*}
On any interval $x \in [-M, M]$, we have 
\begin{equation*}
  \abs{
    g'_n(x) - g(x)
  } = \abs{
    \frac{1}{2} + \frac{x}{n} - \frac{1}{2}
  } = \abs{
    \frac{x}{n}
  } \leq \frac{M}{n}
\end{equation*}
so a choice of $n > \frac{M}{\epsilon}$ will suffice.

To show that $g'(x) = \lim g'_n(x)$, we need 
\begin{itemize}
  \item $g_n$ be a sequence of differentiable functions, yes we have this.
  \item $g'_n$ converge uniformly to $g$, yes.
  \item $x_0 \in [M, -M]$ such that $g_n(x_0)$ is convergent.
  We can choose $x = 0$, then $g_n(0) \to 0 = g(0)$.
\end{itemize}

\item algebraically calculating the limit,
\begin{equation*}
  f(x) = \lim_{n\to\infty} \frac{
    nx^2 + 1
  }{2n+x} 
  = \frac{x^2}{2}
\end{equation*}

So therefore 
\begin{equation*}
  f'(x) = 2x/2 = x
\end{equation*}

Now, 
\begin{equation*}
  f'_n(x) = \frac{
    (2nx)(2n+x) - (nx^2+1)(1)
  }{
    \pa{2n+x}^2
  } = \frac{
    4n^2x + nx^2 - 1
  }{
    4n^2 + 4nx + x^2
  }
\end{equation*}
To show uniform convergence,
\begin{align*}
  \abs{
    \frac{
      4n^2x + nx^2 - 1
    }{
      4n^2 + 4nx + x^2
    } - x
  } 
  &= \abs{
    \frac{
      4n^2x + nx^2 - 1
    }{
      4n^2 + 4nx + x^2
    } -
    \frac{
      4n^2x + 4nx^2 + x^3
    }{
      4n^2 + 4nx + x^2
    }
  }\\
  &= \abs{
    \frac{
      -3nx^2 - x^3 - 1
    }{
      4n^2 + 4nx + x^2
    }
  }\\
  &\leq \frac{
    \abs{3nx^2} + \abs{x^3} + \abs{1}
  }{
    \abs{
      4n^2 + 4nx + x^2
    }
  }\tag{Triangle Inequality}\\
  &\leq \frac{
    \abs{3nx^2} + \abs{x^3} + \abs{1}
  }{
    \abs{
      4n^2 - 4nM
    }
  }\tag{Making the denominator smaller}\\
  &\leq \frac{
    3nM^2 + M^3 + 1
  }{
    \abs{
      4n^2 - 4nM
    }
  }
\end{align*}
Since $n \to \infty$ makes this expression $\to 1/n$, 
we can make this quantity as small as we want, to be 
smaller than any $\epsilon > 0$.

To summarize, we have 
\begin{itemize}
  \item $f_n$ all differentiable
  \item $f'_n \to x = f'$ uniformly
  \item $f_n$ is convergent for any $x$.
\end{itemize}
}
}

\bx{
We want to show for any $x \in [a, b]$ that 
\begin{equation*}
  \abs{
    f_n(x) - f(x)
  } < \epsilon.
\end{equation*}
We can try to prove the Cauchy Criterion instead, so 
\begin{align*}
  \abs{f_n(x) - f_m(x)}
  &= \abs{
    f_n(x) - f_m - \pa{
      f_n(x_0) - f_m(x_0)
    } + \pa{
      f_n(x_0) - f_m(x_0)
    } 
  }\\
  &= \abs{
    f_n(x) - f_m - \pa{
      f_n(x_0) - f_m(x_0)
    }
  } + \abs{
    f_n(x_0) - f_m(x_0)
  }\\
  &= \abs{
    \pa{
      f_n(x) - f_n(x_0)
    } - \pa{
      f_m(x) - f_m(x_0)
    }
  } + \abs{
    f_n(x_0) - f_m(x_0)
  }\\
\end{align*}
Now, we first choose $N_1$ such that for $m, n \geq N_1$,
\begin{equation*}
  \abs{
    f_n(x_0) - f_m(x_0)
  } < 
  \frac{
    \epsilon
  }{2} 
\end{equation*}
Now, we apply MVT to 
\begin{align*}
  \abs{
    \pa{
      f_n(x) - f_n(x_0)
    } - \pa{
      f_m(x) - f_m(x_0)
    }
  }
  &=
  \abs{x-x_0}
  \abs{
    \pa{
      \frac{f_n(x) - f_n(x_0)}{x-x_0}
    } - \pa{
      \frac{f_m(x) - f_m(x_0)}{x-x_0}
    }
  }\\
  &= M
  \abs{
    f'_n(c_1) - f'_n(c_2)
  }\tag{$\abs{x-x_0}$ bounded since $x, x_0 \in [a, b]$}
\end{align*}
By MVT, $c_1, c_2 \in [a, b]$.
Since we know that $f'_n$ converges uniformly on $[a, b]$,
we know we can find $N_2$ such that 
\begin{equation*}
  \abs{
    f'_n(c_1) - f'_n(c_2)
  } < \frac{
    \epsilon
  }{
    2M
  }
\end{equation*}
Putting everything together, if we choose $N = \max(N_1, N_2)$,
we can show that for $m, n \geq N$, that
\begin{equation*}
  \abs{f_n(x) - f_m(x)} = 
  \abs{
    \pa{
      f_n(x) - f_n(x_0)
    } - \pa{
      f_m(x) - f_m(x_0)
    }
  } + \abs{
    f_n(x_0) - f_m(x_0)
  } < \frac{\epsilon}{2} + \frac{\epsilon}{2} = \epsilon
\end{equation*}
}