%% 6.4 %%

\section{Series of Functions}
\setcounter{exercise}{0}

\bx{
\AFSOC $g_n \to \epsilon_0 \neq 0$.
Then we can find some $N$ where $\abs{g_n - \epsilon_0} < \epsilon_0/2$.
WLOG $\epsilon_0 > 0$, then we will be summing 
an infinite number of positive numbers, which diverges to $+\infty$.
}

\bx{
WLOG $k_1 < k_2$,
\begin{align*}
  \abs{
    s_{k_1} - s_{k_2}
  } 
  &\leq 
  \abs{
    \sum_{n=k_1+1}^{k_2} f_n(x)
  }\\
  &\leq 
  \sum_{n=k_1+1}^{k_2} \abs{f_n(x)}\\
  &\leq 
  \sum_{n=k_1+1}^{k_2} M_n \tag{$\abs{f_n(x)} < M_n$}\\
  &\leq 
  \abs{
    \sum_{n=k_1+1}^{k_2} M_n
  } < \epsilon
\end{align*}
The last part follows, since we know $\sum_{n=1}^\infty M_n$ converges,
so we can find this $N$ such that for $k_1, k_2 \geq $, we have convergence.
}

\bx{
\ea{
\item WLOG $k_1 < k_2$, then we have 
\begin{align*}
  \abs{g_{k_1}(x) - g_{k_2}(x)} 
  &=
  \abs{
    \sum_{n=1}^{k_1} \frac{\cos(2^nx)}{2^n} -
    \sum_{n=1}^{k_2} \frac{\cos(2^nx)}{2^n}
  }\\
  &= \abs{
    \sum_{n=k_1+2}^{k_2} \frac{\cos(2^nx)}{2^n}
  }\\
  &\leq
  \abs{
    \sum_{n=k_1+2}^{k_2} \frac{1}{2^n}
  }\\
  &\leq
  \abs{
    \sum_{n=k_1+2}^\infty \frac{1}{2^n}
  }\\
  &\leq \abs{
    \frac{1}{2^{k_1-1}}
  }
\end{align*}
So we just need to choose $N > \log_2\pa{\frac{2}{\epsilon}}$
Since $g(x)$ converges uniformly, and each 
\begin{equation*}
  f_n = \frac{\cos(2^nx)}{2^n}
\end{equation*}
is continuous, since we can find a derivative that is defined everywhere,
we conclude that $g(x)$ is continuous.A

Ok so apparently I did way more work than you had to,
you can use 
\begin{equation*}
  M_n = \frac{1}{2^n} \geq \abs{\cos(2^nx)/2^n}
\end{equation*}
and the Weierstrass M-Test to prove uniform convergence too...

\item Each 
\begin{equation*}
  f_n = x^n/n^2 \Rightarrow f'_n = x^{n-1}/n
\end{equation*}
has a derivative that exists, so we conclude each $f_n$ is continuous.
We just need to show that $h(x)$ converges uniformly.

Define 
\begin{equation*}
  M_n = \frac{1}{n^2} \geq \abs{
    \frac{x^n}{n^2}
  }
\end{equation*}
This is true since $\abs{x^n} \leq 1$ since 
$x \in [-1, 1]$.
Now, since $\sum_{n=1}^\infty 1/n^2$ converges,
we can use the Weierstrass M-Test to conclude that $h(x)$ converges uniformly.
}
}

\bx{
We can show that 
\begin{equation*}
  M_n = \frac{1}{2^n} \geq \abs{
    \frac{h(2^nx)}{2^n}
  }
\end{equation*}
We know that $\sum_{n=1}^\infty 1/2^n$ converges. Therefore, $g(x)$ 
converges uniformly.

Finally, each $h(2^nx)/2^n$ is continuous, so we can conclude 
that $g(x)$ is therefore continuous as well.
}

\bx{
\ea{
\item We have 
\begin{equation*}
  f'_n(x) =
  \frac{
    \cos(kx)
  }{k^2}
\end{equation*}
We can show this converges uniformly to 0,
\begin{equation*}
  \abs{
    \frac{
      \cos(kx)
    }{k^2}
  } \leq \frac{1}{k^2}
\end{equation*}

We can show that $f(x) \to 0$ for $x = 2\pi$, since $\sin(k 2\pi) = 0$.
Therefore, we can conclude, since $\sum f'_n$ converges uniformly,
that $f(x)$ is differentiable.

Since each $f'_n$ is continuous, and $\sum f'_n \to f'$ uniformly,
we can conclude that $f'$ is continuous.

\item No, since we can only do 
\begin{equation*}
  \frac{1}{n} \geq \abs{-\sin(nx)/n},
\end{equation*}
but we know the Harmonic series does not converge.
}
}

\bx{
For any $x_0 \in (0, 1)$ fixed, choose $y \in (x_0, 1)$.
Then we can show 
\begin{align*}
  1 &\leq y^0\\
  x &\leq y\\
  x^2 &\leq y^2 \Rightarrow
  \frac{x^2}{2} \leq y^2\\
  x^3 &\leq y^3 \Rightarrow
  \frac{x^3}{3} \leq y^3
\end{align*}
which means for $M_n = y^n$, we can show $\sum M_n \to \frac{1}{1-y}$,
which exists when $y \in (0, 1)$, and therefore 
$f(x)$ converges uniformly.

We know each $\pa{\frac{x^n}{n}}^{'} = x^{n-1}$,
so each $f_n$ is continuous.

Therefore, we conclude $f(x)$ is continuous at $x_0$.
}

\bx{
\ea{
\item We have 
\begin{equation*}
  M_n = 1/n^2 \geq h_n(x),
\end{equation*}
so $h(x)$ converges uniformly.
Each 
\begin{equation*}
  h'_n(x) = \frac{
    -2x
  }{
    \pa{x^2 + n^2}^2
  }
\end{equation*}
exists for all $x$, so they are all continuous.
Therefore, we conclude $h$ is continuous on all of $\mathbb{R}$.

\item We can show that for $x \in \pbra{-M, M}$, that 
\begin{equation*}
  \abs{
    \frac{
      -2x
    }{
      \pa{x^2 + n^2}^2
    }
  } \leq \frac{
    2M
  }{n^4} 
\end{equation*}
Since $\sum 2M/n^4$ converges, we can conclude with the Weierstrass M-Test
that $h'_n$ converges uniformly.
Therefore, $h$ is differentiable, and also $h'$ is continuous from uniform convergence.

Since we can choose any $M$ for our argument, we conclude that $h$ is 
differentiable and continuous on all of $\mathbb{R}$.
}
}

\bx{
For $x \not\in \mathbb{Q}$, we can show 
$u_n(x)$ is continuous, since for $x < r_n$, we can choose
a small enough $\delta$ such that $u_n(y) = 0$ for $y \in V_\delta(x)$.
Similar reasoning applies to $x > r_n$.
We can then show $h(x)$ converges uniformly, by the Weierstrass M-Test
since 
\begin{equation*}
  M_n = \frac{1}{2^n} \geq u_n(x)
\end{equation*}
Therefore, since $u_n$ are all continuous, and $h$ converges uniformly,
we conclude that $h$ is continuous.

$h$ is monotone, since every $u_n(x)$ is increasing, so 
for $x < y$,
\begin{align*}
  \forall n u_n(x) &\leq u_n(y)\\
  \sum_{n=1}^k u_n(x) &\leq \sum_{n=1}^k u_n(y)\\
  \lim_k \sum_{n=1}^k u_n(x) &\leq \lim_k \sum_{n=1}^k u_n(y)\\
  h(x) &\leq h(y) \tag{$h$ converges uniformly}
\end{align*}

\label{chap6:ex:_rational_monotone_continuous}
}