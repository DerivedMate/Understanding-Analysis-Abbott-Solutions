%% 2.8 %%
\section{Double Summations and Products of Infinite Series}
\setcounter{exercise}{0}

\bx{
\begin{align*}
	\lim s_{nn} 
	&= -1 + -\frac{1}{2} - \frac{1}{4} - \cdots\\
	&= -\sum_{i=0}^{\infty} \pa{\frac{1}{2}}^i \\
	&= -2.
\end{align*}
The value is equal to summing column-wise.
}

\bx{
By the Absolute Convergence test, since we know for fixed $i$ that $\sum_{j=1}^\infty \abs{a_{ij}}$ converges, then we know for fixed $i$ that each $\sum_{j=1}^\infty a_{ij}$ converges to some $c_i$ as well. 

Then, since 
\begin{align*}
	\sum_{j=1}^\infty \abs{a_{ij}} 
	&\geq \abs{
		\sum_{j=1}^\infty a_{ij}
	} \\
	\Rightarrow b_i 
	&\geq \abs{c_i}\\
	\abs{b_i} &\geq \abs{c_i},
\end{align*}
and we know that $\sum_{i=1}^\infty b_i$ converges, 
we conclude that $\sum_{i=1}^\infty c_i$ must converge as well by the 
Absolute Convergence test,
implying that 
\begin{equation}
	\sum_{i=1}^\infty \sum_{j=1}^\infty a_{ij}
\end{equation}
converges as well.
}

\bx{
\ea{
\item Since $\sum_{i=1}^\infty \sum_{j=1}^\infty \abs{a_{ij}}$ converges, 
we have that 
\begin{equation*}
	t_{mn} \leq \sum_{i=1}^\infty \sum_{j=1}^\infty \abs{a_{ij}} = L
\end{equation*}
Since $t_{nn}$ is an increasing sequence, and is bounded above, 
by the Monotone Convergence Theorem, $t_{nn}$ converges.

\item For any $\epsilon > 0$, $\exists N : n > m \geq N$ such that 
$\abs{t_{nn} - t_{mm}} < \epsilon$. Now, consider
\begin{align*}
	\abs{s_{n+1, n+1} - s_{nn}}
	&\leq \abs{t_{n+1, n+1} - t_{nn}} \\
	&< \epsilon.
\end{align*}
So $(s_{nn})$ is a Cauchy Sequence and converges.
}
}

\bx{
\ea{
	\item Since we know there exists a $t_{m_0n_0}$ such that 
	$t_{n_0n_0} > B - \frac{\epsilon}{2}$, and $t_{nn}$ is increasing 
	and that $B$ is an upper bound, we can conclude that for 
	$N_1 = \max\{m_0, n_0\} : m, n \geq N_1$,
	\begin{equation}
			B - \frac{\epsilon}{2} < t_{mn} \leq B. 
	\end{equation}
	\label{chap2:part_tmn_converge}

	\item For any $\epsilon > 0$,
	since $(t_{mn})$ is bounded above by  $A = \sum_{i=1}^\infty \sum_{j=1}^\infty \abs{a_{ij}}$, 
	from part (\ref{chap2:part_tmn_converge}) we can choose $N : m, n \geq N$ such that 
	\begin{align*}
		A + \frac{\epsilon}{2} < \,\,&t_{mn} < A + \epsilon\\
		\Rightarrow \frac{\epsilon}{2} < \,\,&\abs{t_{mn} - A} < \epsilon\\
		&\abs{t_{mn} - \sum_{i=1}^\infty \sum_{j=1}^\infty \abs{a_{ij}}} < \epsilon.
	\end{align*}
	
	Then, we can see that this $N$ also works to show
	\begin{align*}
	\abs{s_{mn} - S} 
	&= \abs{s_{mn} - \sum_{i=1}^\infty \sum_{j=1}^\infty a_{ij}} \\
	&= \abs{\sum_{i=m+1}^\infty \sum_{j=n+1}^\infty a_{ij}} \\
	&< \abs{\sum_{i=m+1}^\infty \sum_{j=n+1}^\infty \abs{a_{ij}}} \\
	&= \abs{t_{mn} - \sum_{i=1}^\infty \sum_{j=1}^\infty \abs{a_{ij}}} \\
	&< \epsilon.
	\end{align*}
}
}

\bx{
We know $\lim_{n\to\infty} \sum_{j=1}^n a_{ij} = r_i$, so
for any $\epsilon > 0$, $\exists N : n \geq N$ such that 
\begin{equation*}
	\abs{
		\sum_{j=1}^n a_{ij} - r_i
	} < \frac{\epsilon}{m}.
\end{equation*}
if we fix $m \geq N$.

Then 
\begin{align*}
	\abs{
		(r_1 + r_2 + \cdots + r_m) - S
	} 
	&= \abs{
		\sum_{i=1}^m \pa{r_i - \sum_{j=1}^n a_{ij}}
	}\\
	&\leq \sum_{i=1}^m \abs{r_i - \sum_{j=1}^n a_{ij}}\\
	&\leq m \cdot \frac{\epsilon}{m} = \epsilon
\end{align*}

Therefore, we conclude that $\sum_{i=1}^\infty \sum_{j=1}^\infty a_{ij}$ converges to $S$.

\TODO not sure where I have to use the Order Limit Theorem...
\label{chap2:ex_double_sum_converge}
}

\bx{
For $\sum_{i=1}^\infty \sum_{j=1}^\infty a_{ij}$, the proof is essentially 
the same as Exercise \ref{chap2:ex_double_sum_converge}
to show it converges to $S$, except we fix $n$ this time instead of $m$.
}

\bx{
\ea{
\item Define $t_{nn} = \sum_{i=1}^n \sum_{j=1}^n \abs{a_{ij}}$.
Also define $u_{n} = \sum_{k=2}^n \abs{d_k}$
Then we know for $n \geq 2$, 
\begin{equation*}
	u_n \leq t_{nn} = L \tag{$t_{nn}$ converges}
\end{equation*}
Since $u_n$ is an increasing sequence and is bounded above, we conclude 
from the Monotone Convergence Theorem that $u_n = \sum_{k=2}^n \abs{d_k}$
also converges. Then, $ \sum_{k=2}^n d_k$ converges absolutely.

\item We need to bound $\abs{d_k - S}$ somehow. Consider the following diagram
\begin{figure}[H]
	\centering
	\def\pictureWidth{2}
	\begin{tikzpicture}
		\filldraw[pattern=north east lines] 
			(0, \pictureWidth) rectangle (\pictureWidth, 2*\pictureWidth) 
				node[pos=0.5] {\colorbox{white}{$t_{\frac{k}{2}\frac{k}{2}}$}}
		;
		\filldraw[pattern=horizontal lines]
			(0, 0) -- 
			(0, \pictureWidth) -- 
			(\pictureWidth, \pictureWidth) -- cycle
		;
		\filldraw[pattern=horizontal lines]
			(\pictureWidth, \pictureWidth) -- 
			(\pictureWidth, 2*\pictureWidth) -- 
			(2*\pictureWidth, 2*\pictureWidth) --  cycle
		;

		\draw 
			(2*\pictureWidth, 2*\pictureWidth) -- (3*\pictureWidth, 2*\pictureWidth)
			(0, 0) -- (0, -\pictureWidth)
		;

		\draw 
			(\pictureWidth, 2*\pictureWidth) node[above] {$k/2$}
			(0, \pictureWidth) node[left] {$k/2$}
			(\pictureWidth, \pictureWidth) node[below right] {$a_{\frac{k}{2}\frac{k}{2}}$}
		;

		\node[circle, fill=black, minimum size=1pt] (akk) at (\pictureWidth, \pictureWidth) {};
	\end{tikzpicture}
	\caption{Demonstrating how we bound our sum}
	\label{chap2:fig:bounding_dkk}
\end{figure}
We know that for any $\epsilon > 0$,
we can choose $N : n \geq N$ so that 
\begin{equation}
	\abs{t_{nn} - S} < \epsilon
	\label{chap2:eq:tnn_eps}
\end{equation}
Now, choose $N_1 = 2N$.
We can use Figure \ref{chap2:fig:bounding_dkk} 
to see that for $k \geq N_1$,
\begin{align*}
	\abs{d_{kk} - S} 
	&\leq \abs{t_{\frac{k}{2}\frac{k}{2}} - S}\\
	&< \epsilon \tag{From Equation (\ref{chap2:eq:tnn_eps})}
\end{align*}
Therefore, we can conclude $\sum_{k=2}^\infty d_k$ converges to $S$.
}
}

\bx{
\ea{
\item See that 
\begin{align*}
	AB 
	&\geq \pa{\sum_{i=1}^\infty \abs{a_i}} \pa{\sum_{j=1}^\infty \abs{b_j}}\\
	&= \sum_{i=1}^\infty \pa{\abs{a_i}\sum_{j=1}^\infty \abs{b_j}} \tag{$\sum_{j=1}^\infty \abs{b_j}$ is constant wrt $i$}\\
	&= \sum_{i=1}^\infty \pa{\sum_{j=1}^\infty \abs{a_i}\abs{b_j}}\\
	&= \sum_{i=1}^\infty \sum_{j=1}^\infty \abs{a_ib_j}
\end{align*}
Since $\sum_{i=1}^m\sum_{j=1}^n \abs{a_ib_j}$ is bounded, and the partial sums 
$s'_{nn} = \sum_{i=1}^n\sum_{j=1}^n \abs{a_ib_j}$ are increasing, we can conclude 
$\sum_{i=1}^\infty\sum_{j=1}^\infty \abs{a_ib_j}$ converges by the Monotone Convergence Theorem.

\item Let $s^a_n, s^b_n$ be the partial sums of $(a_n), (b_n)$ respectively.
Then,
\begin{align*}
	\lim_{n\to\infty} s_{nn}
	&= \lim_{n\to\infty} \sum_{i=1}^n\sum_{j=1}^n a_ib_j\\
	&= \lim_{n\to\infty} \sum_{i=1}^n \pa{a_i \sum_{j=1}^n b_j}\\
	&= \lim_{n\to\infty} \pa{\sum_{i=1}^n a_i} \pa{\sum_{j=1}^n b_j}\\
	&= \lim_{n\to\infty} s^a_n s^b_n\\
	&= AB
\end{align*}
Therefore, by Theorem 2.8.1, we can conclude that 
\begin{equation*}
	\sum_{i=1}^\infty\sum_{j=1}^\infty a_ib_j
	= \sum_{j=1}^\infty\sum_{i=1}^\infty a_ib_j
	= \sum_{i=2}^\infty d_k
	= AB
\end{equation*}
}
}