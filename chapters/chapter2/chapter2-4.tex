%% 2.4 %%
\section{The Monotone Convergence Theorem and a First Look at Infinite Series}
\setcounter{exercise}{0}

\bx{
Suppose $\sum_{n=0}^\infty 2^nb_{2^n}$ diverges. 
Fix $m, k$ so that $m \geq 2^{k+1} - 1$, then
\begin{align*}
	\sum_{i=1}^{m} b_i 
	&\geq \sum_{i=1}^{2^{k+1}-1} b_i\\
	& = s_{2^{k+1} - 1}\\
	&= t_k
\end{align*}
Since $t_k$ is a diverging sequence, then $b_m$ will also diverge.
}

\bx{ 
\ea{
\item We can show by induction that the sequence is decreasing. 
Thus, because the sequence starts at 3, we know it is bounded below by 0. 
Thus, the sequence converges.
\item If $\lim x_n$ exists, then  $\lim x_{n+1}$ must be the same limit, because if the limit is a different value or doesn't exist, then $(x_n)$ does not converge.
\item Suppose $\lim x_n = \lim x_{n+1} = x$. Then 
\begin{align*}
	x &= \frac{1}{4-x} \\
	x^2 -4x + 1 &= 0 \\
	\lra x &= 2 - \sqrt{3}
\end{align*}
The other root is too large and does not work with the initial conditions.
}
}

\bx{
We can use induction to show that $(y_n)$ is increasing. 
Since the sequence is increasing and starts at 1, 
we know that $(y_n)$ is bounded above by 4 and below by 0. 
Thus, by the Monotone Convergence Theorem, we conclude that $(y_n)$ converges. Now, we find the limit of the recurrence by taking the limits of both sides of the equation,
\begin{align*}
	y &= 4 - \frac{1}{y} \\
	y^2 - 4y + 1 &= 0 \\
	y &= 2 + \sqrt{3}
\end{align*}
}

\bx{
We can define the recurrence of this sequence as 
\begin{equation}
	a_{n+1} = \sqrt{2 a_{n}}.
\end{equation}
We can prove by induction that this sequence is increasing. 
We can also bound the sequence since this sequence can also be viewed as 
\begin{equation*}
	2^{\frac{1}{2}}, 
	2^{\frac{1}{2} + \frac{1}{4}}, 
	2^{\frac{1}{2} + \frac{1}{4} + \frac{1}{8}},
	\cdots
\end{equation*}
You can take the infinite sum $\sum_{i=1}^{\infty} 2^{-i} = 1$ and get $2^1 = 2$ as your final answer.

The other way to solve this problem is to look at the limits of $x_n, x_{n+1}$,
which must be equal. Let's say their limit is $x$, then
\begin{align*}
	x_{n+1} &= \sqrt{2x_{n}}
	x &= \sqrt{2x} \\
	x^2 - 2x &= 0\\
	x &= 2 \tag{from $x_0 = 1$}.
\end{align*}
}

\bx{
\ea{
	\item By induction, we have
	
	\textbf{Base Case}: $x_1 = 2 \lra x_1^2 = 4 \geq 2$.
	
	\textbf{Inductive Hypothesis}: Given that for some $x_n, x_n^2 \geq 2$.
	
	\textbf{Inductive Step}: Consider
	\begin{align*}
		x_{n+1}^2 &= \frac{1}{4}\pa{x_n^2 + 4 + \frac{4}{x_n^2}} \\
		&\geq \frac{1}{4}\pa{2 + 4 + 4/2} = 2.
	\end{align*}
	Therefore, we conclude $\forall n, x_n \geq 2$.

	Now we can show
	\begin{align*}
		x_n - x_{n+1} &= x_n - \frac{1}{2}\pa{x_n + \frac{2}{x_n}} \\
		&= \frac{\frac{1}{2}x_n^2-1}{x_n} \\
		&\geq 0,
	\end{align*}
	which means the sequence is decreasing, so by the Monotone Convergence Theorem we know that $(x_n)$ converges. 
	We now take limits of $x$ on both sides of the recurrence, yielding,
	\begin{align*}
		x &= \frac{1}{2}\pa{x + \frac{2}{x}} \\
		\frac{1}{2}x - \frac{1}{x} &= 0 \\
		x^2 - 2 &= 0 \\
		\lra x &= \sqrt{2}.
	\end{align*}
	\item We can modify the sequence to converge to $\sqrt{c}, c\geq 0$ by setting $x_1 = c$, and
	\begin{equation}
		x_{n+1} = \frac{1}{c}\pa{(c-1)x_n + \frac{c}{x_n}}
	\end{equation}
}
}

\bx{
\ea{
	\item Since we know that $(a_n)$ is bounded, it must also be the case that $\sup (a_n)$ is bounded.
	Then, $\sup\{a_k\}$ is a decreasing sequence, so by the Monotone Convergence Theorem, 
	we know that $(y_n)$ converges.
	\item We can define 
	\begin{align}
		\lim \inf a_n &= \lim z_n, \text{ where } \\
		\lim z_n &= \inf\{a_k\,:\,k\geq n\}.
	\end{align}
	Since $\inf\{a_k\}$ is a increasing sequence, and $(a_n)$ is bounded, we know it converges.
	\item For any set $A$, $\inf A \leq \sup A$, so $\forall n, \inf \{a_k:k\geq n\} \leq \sup \{a_k:k\geq n\}$.
	
	An example when the inequality is strict is 
	\begin{equation}
		a_n = (-1)^n,
	\end{equation}
	since $\lim\inf a_n = -1, \lim\sup a_n = 1$.
	
	\item ($\Rightarrow$) 
	Suppose 
	\begin{equation}
		\lim \inf a_n = \lim \sup a_n = L,
	\end{equation}
	then given some $\epsilon > 0$,
	we know $\exists N: n \geq N$ so that, define $A_n = \{a_k:k \geq n\}$
	\begin{align*}
		\abs{
			\inf A_n - L
		} &< \epsilon\\
		\abs{
			\sup A_n - L
		} &< \epsilon
	\end{align*}
	since every element $k\geq n, \inf A_n \leq a_k \leq \sup A_n$, we conclude 
	\begin{equation*}
		k \geq n \geq N \quad \abs{
			a_k - L
		} < \epsilon,
	\end{equation*}
	so $\lim a_n = L$.

	($\Leftarrow$)
	Suppose 
	\begin{equation*}
		\lim a_n = L,
	\end{equation*}
	then given some $\epsilon > 0$, we know $\exists N: n \geq N$ so thats
	\begin{equation*}
		\abs{
			a_n - L
		} < \epsilon/2
	\end{equation*}
	This means every element after $a_n$ lives in this $\epsilon/2$-neighborhood of $L$.
	Now, $\sup A_n$ must be arbitrarily close to the largest element of $A_n$, so we can make
	this distance $\epsilon/2$. That means 
	\begin{equation*}
		\abs{
			\sup A_n - L
		} = \abs{\max\{A_n\} + \epsilon/2 - L} < \epsilon,
	\end{equation*}
	which means $\lim\sup a_n = L$. This is similar for $\inf$.
}
}