% Chapter 1: The Real Number
% Introduces some definitions that lead to the real numbers
% being defined

\chapter{The Real Numbers}

\setcounter{section}{1}

%%% 1.2 %%%
\section{Some Preliminaries}
\bx{
	\ea{
		\item 
		\begin{proof}
			AFSOC $\sqrt{3}$ is rational, so $\exists m, n \in \mathbb{Z}$ such that 
			\begin{equation*}
				\sqrt{3} = \frac{m}{n},
			\end{equation*}
			where $\frac{m}{n}$ is in lowest reduced terms.
			Then we can square both sides, yielding $3 = \pa{\frac{m}{n}}^2 \lra 3n^2 = m^2$. Now, we know $m^2$ is a multiple of 3 and thus $m$ must also. Then, we can write $m = 3k$, and derive
			\begin{align*}
			(\sqrt{3})^2 &= \pa{\frac{3k}{n}}^2 \\
			3n^2 &= 9k^2 \\
			n^2 &= 3k^2
			\end{align*}
			Similar to before, we come to the conclusion that $n$ is a multiple of 3. However, this is a contradiction since $m, n$ are both multiples of 3 and we assumed $\frac{m}{n}$ was in lowest terms. Thus, we conclude $\sqrt{3}$ is irrational.
		\end{proof}
		The same proof for $\sqrt{3}$ works for $\sqrt{6}$ as well.

		\item We cannot conclude that $\sqrt{4} = \frac{m}{n}$ implies that 
		$m$ is a multiple of $4$, since we have 
		\begin{equation*}
			4n^2 = m^2 \quad \Rightarrow \quad 2n = m,
		\end{equation*}
		so we cannot reach our contradiction that $m/n$ is not in lowest terms.
	}
}


\bx{
	\ea{
		\item False. Consider 
		\begin{equation*}
			A_n = \left[0, \frac{1}{n}\right).
		\end{equation*}
		Then 
		\begin{equation*}
			\bigcap_{n=1}^\infty A_n = \{0\}.
		\end{equation*}
		\item True. Since $\forall i, A_i \subseteq A_1$, $\exists x$ such that $\forall i, x \in A_i$.
		Therefore, the intersection cannot be empty. Then, every set is finite, and the intersection of 
		any number of finite sets will be finite.
		\item False. Consider $A = \{1, 2\}, B = \{1\}, C = \{2, 3\}$.
		\begin{equation*}
			\{1, 2\} \cap \pa{
				\{1\} \cup  \{2, 3\}
			} = \{1, 2\}
			\neq \pa{
				\{1, 2\} \cap \{1\}
			} \cup \{2, 3\}
			= \{1, 2, 3\}
		\end{equation*}
		\item True. Intersection is associative.
		\item True. Intersection is distributive over union.
		\begin{proof}
			We will prove 
			\begin{equation}
				A \cap \pa{B \cup C} = \pa{A \cap B} \cup \pa{A \cap C}
				\label{eq:set_distributive}
			\end{equation}
			by set inclusion. 

			\begin{itemize}
				\item Suppose $x \in A \cap \pa{B \cup C}$. By the definition of intersection,
				we know $x \in A$ and $x \in B \cup C$, the latter which means $x \in B$ or $x \in C$.
				
				We can consider 2 cases for $x$,

				\begin{enumerate}[label=\arabic*.]
					\item $x \in B$. Then we know  $x \in A$ and $x \in B$, so $x \in A \cap B$ and therefore $x \in \pa{A \cap B} \cup \pa{A \cap C}$ 
					\item $x \in C$. Symmetric to the case above.
				\end{enumerate}

				in all cases, we see $x \in A \cap \pa{B \cup C}$ implies $x \in \pa{A \cap B} \cup \pa{A \cap C}$, so 
				\begin{equation*}
					A \cap \pa{B \cup C} \subseteq \pa{A \cap B} \cup \pa{A \cap C}
				\end{equation*}

				\item Suppose $x \in \pa{A \cap B} \cup \pa{A \cap C}$.
				Then we have two cases
				\begin{enumerate}[label=\arabic*.]
					\item $x \in A \cap B$. This means $x \in A$ and $x \in B$. If $x \in B$, then $x \in B \cup C$, since $B \subseteq B \cup C$.
					Putting these facts together, we see $x \in A \cap \pa{B \cup C}$.
					\item $x \in A \cap C$. Symmetric to the case above.
				\end{enumerate}
				in all cases, we see $x \in \pa{A \cap B} \cup \pa{A \cap C}$ implies $x \in A \cap \pa{B \cup C}$, so 
				\begin{equation*}
					\pa{A \cap B} \cup \pa{A \cap C} \subseteq A \cap \pa{B \cup C} 
				\end{equation*}
			\end{itemize}
		\end{proof}
	}
}

\bx{
		\ea{
			\item If $x \in \pa{A \cap B}^c$, then we have cases
			\begin{itemize}
				\item $x \in B$ and $x \not\in A$. Then $x \not\in A$ implies $x \in A^c \Rightarrow x \in A^c \cup B^c$.
				\item $x \in A$. Symmetric to above.
				\item $x \not\in A$ and $x \not\in B$. Then $x \in A^c$ so $x \in A^c \cup B^c$.
			\end{itemize}
			
			\item If $x \in A^c \cup B^c$, then we have cases
			\begin{itemize}
				\item $x \in A^c$. Then $x \not\in A$ so $x$ cannot be in the intersection of $A$ and $B$, so 
				$x \in \pa{A \cap B}^c$.
				\item $x \in B^c$. Symmetric to above.
			\end{itemize}
			\item Proof for $\pa{A \cup B}^c = A^c \cap B^c$ pretty similar to above.
		}
}

\bx{
	We are verifying the triangle inequality with $a, b$.
	\ea{
		\item If $a, b$ have the same sign, then 
		\begin{align*}
			\abs{a + b} &= a+b\\
			\abs{a} + \abs{b} &= a + b\\
			\Rightarrow\, \abs{a+b} &= \abs{a} + \abs{b}\\
			\Rightarrow\, \abs{a+b} &\leq \abs{a} + \abs{b}
		\end{align*}

		\label{ex_triangle_ineq_same_sign}

		\item 
		\begin{itemize}
			\item $a \geq 0, b < 0$.
			\begin{align*}
				\abs{a + b} 
				&\leq \abs{a}\\
				&\leq \abs{a} + \abs{b}
			\end{align*}

			\item $a + b \geq 0$.
			At most one of $a, b$ is negative. If they are both positive, then 
			we have already shown this in part \ref{ex_triangle_ineq_same_sign}.
			Otherwise, WLOG $a$ is negative. Then 
			\begin{align*}
				\abs{a + b} 
				&\leq \abs{b}\\
				&\leq \abs{a} + \abs{b}
			\end{align*}
		\end{itemize}
	}
}

\bx{
	\ea{
		\item Substitute in $b' = -b$ into the triangle inequality.
		\item Easy to prove directly without using triangle inequality. \TODO.
		
		A direct proof will look something like:
		\begin{itemize}
			\item If $a, b$ are the same sign, then equality holds 
			\item If $a, b$ are different signs, then if $b$ is negative, then $\abs{a-b} = \abs{a} + \abs{b}$, 
			and if $a$ is negative, then $\abs{a-b} = \abs{a} + \abs{b}$, both of which bound $\abs{\abs{a} - \abs{b}}$.
		\end{itemize}
	}
}

\bx{
	\ea{
		\item Yes, since $f\pa{A \cap B} = [1, 4] = [0, 4] \cap [1, 16] = f(A) \cap f(B)$. This is by coincidence though, as we will later see.
		Yes, since $f\pa{A \cup B} = [0, 16] = [0, 4] \cup [1, 16] = f(A) \cup f(B)$.
		\item Choose $A = [-2, 0], B = [0, 2]$
		\item Suppose $x \in g(A\cap B)$, then $\exists x' \in A \cap B$ such that $g(x') = x$.
		Since $x' \in A$ and $x' \in B$, we know $x = g(x') \in g(A), g(B)$, so we conclude $x \in g(A) \cap g(B)$.
		\item Equality. \TODO too lazy to write out the proof. Similar to above.
	}
}

\bx{
	\ea{
		\item \TODO I don't think we want to include $x \in \mathbb{I}$...
		\begin{align}
			f^{-1}(A) &= [0, 2]\\
			f^{-1}(B) &= [0, 1]
		\end{align}
		We see $f^{-1}(A\cap B) = f^{-1}(A) \cap f^{-1}(B)$ in this case.
		$f^{-1}(A\cup B) = f^{-1}(A) \cup f^{-1}(B)$ is also true.
		\item \TODO
	}
}

\bx{
	Negating statements. Took some liberties. Also notice that these statements are not necessarily true.
	\ea{
		\item There exists a real number satisfying $a < b$, such that $\forall n \in \mathbb{N}$, $a+1/n \geq b$.
		\item There exists two distinct real numbers such that there is not a rational number between them.
		\item There exists a natural number $n \in \mathbb{N}$ such that $\sqrt{n}$ is not a natural number nor an irrational number.
		\item  There exists a real number $x \in \mathbb{R}$ such that $\forall n \in \mathbb{N}$, $n \leq x$.
	}
}

\bx{
	We are given the sequence 
	\begin{equation}
		x_1 = 1, x_{n+1} = \frac{1}{2}x_{n} + 1
	\end{equation}
	and want to show $\forall i \geq 1, x_i < 2$.

	We can show this with a direct proof of summation.

	An alternative that the book probably wants to see is using \textbf{induction}.

	\begin{itemize}
		\item Base Case: $x_1 = 1 < 2$
		\item Inductive case. Assume $\forall i < n+1, x_i < 2$. Then $x_i/2 + 1 < 2$ since $x_i/2 < 1$.
		\item By induction our original claim is proved.
	\end{itemize}
	\label{ex:induction_1_2_9}
}

\bx{
	\ea{
		\item Similar to Exercise \ref{ex:induction_1_2_9}. $y_n < 4$ means $(3/4)y_n < 3$ so $(3/4)y_n + 1 < 4$
		\item In brief,
		\begin{align}
			y_n &\leq \frac{3}{4}y_n + \frac{1}{4}y_n\\
			&< \frac{3}{4}y_n + 1 \tag{Using $y_n < 4$}\\
			&< y_n+1 \tag{Sequence definition}
		\end{align}
	}
}

\bx{
	A combinatorial argument is that in order to construct a set, we have 2 choices for every element, 
	to include it or not to. Therefore, we have 
	\begin{equation*}
		\prod_{i=1}^n 2 = 2^n
	\end{equation*}
}

\bx{
	\ea{
		\item We know that $\pa{A_1 \cup A_2}^c = A_1^c \cap A_2^c$. So if we are trying to show 
		$\pa{A_1 \cup A_2 \cup A_3}^c = \pa{A_1 \cup A_2}^c \cap A_3^c = A_1^c \cap A_2^c \cap A_3^c$.
		Induction lets us apply the property on smaller parts of our expression.
		\item Induction only proves the property for some $n \in \mathbb{N}$, i.e. some finite $n$. It is not shown for an infinite $n$.
		\item \TODO. Sketch: If $x$ is not in the union of all the $A_n$, then $x$ cannot be part of any particular $A_n$ either, or else it would be in the union.
	}
}

%%% 1.3 %%%
\setcounter{exercise}{0}

\section{The Axiom of Completeness}
\begin{exercise}
\begin{enumerate}[label=(\alph*)]
	\item We compute the additive inverse for each element in $\mathbb{Z}_5$. 
	\begin{align*}
		0 + 0 &\equiv 0 \\
		1 + 4 &\equiv 0 \\
		2 + 3 &\equiv 0 \\
		3 + 2 &\equiv 0 \\
		4 + 1 &\equiv 0
	\end{align*}
	\item We compute the multiplicative inverse for each element in $\mathbb{Z}_5$. 
	\begin{align*}
		1 \times 1 &\equiv 1 \\
		2 \times 3 &\equiv 1 \\
		3 \times 2 &\equiv 1 \\
		4 \times 4 &\equiv 1
	\end{align*}
	\item $\mathbb{Z}_4$ is not a field because multiplicative inverses do not exist for every single element. We conjecture that $\mathbb{Z}_n$ always has additive inverses and only has multiplicative inverses if $n$ is prime.
\end{enumerate}
\end{exercise}

\begin{exercise}
\begin{enumerate}[label=(\alph*)]
	\item  $s = \inf A$ means 
	\begin{enumerate}[label=\roman*)]
		\item $s$ is a lower bound for $A$
		\item if $b$ is any lower bound for $A$, then $b \leq s$
	\end{enumerate}
	\item If $s\in \mathbb{R}$ is a lower bound for $A \subseteq \mathbb{R}$, then $s=\inf A$ iff $\forall \epsilon > 0, \exists a \in A$ such that $s + \epsilon > a$.
	\begin{proof}
		($\Rightarrow$)If $s=\inf A$, then $s$ is the greatest lower bound for $A$, meaning any $s+ \epsilon$ for $\epsilon > 0$ will be greater than some element of $A$, otherwise $s+ \epsilon$ is a greater lower bound and leads to a contradiction that $s \neq \inf A$.
		($\Leftarrow$) If $\forall \epsilon > 0, \exists a \in A$ such that $s+\epsilon > a$, then since $s$ is a lower bound, $\forall b > s$, $b$ will not be a lower bound for $A$ since $b > s \lra \exists a \in A \mid b > a$. Thus, all lower bounds b must be such that $b \leq s$, and we conclude $s = \inf A$.
	\end{proof}
\end{enumerate}
\end{exercise}

\begin{exercise}
\begin{enumerate}[label=(\alph*)]
	\item \TODO
	\item There might be a typo in this question. I think the question was meant to read ``explain why there is no need to assert that the greatest \textit{lower bound} in the Axiom of Completeness.'' In this case, the answer would be that the Axiom of Completeness already implies the greatest lower bound property, so there is no need to explicitly state it.
	\item We can take the negative of all elements in $A$, find $\sup A$, and then negate again to get $\inf A$.
\end{enumerate}
\end{exercise}

\begin{exercise}
	If $B \subseteq A$, then 
	\begin{align*}
		\sup A = s &\geq a \in A \\
			s &\geq b \in B \tag{since $B\subseteq A$} \\
			\lra s &\geq \sup B \tag{since $s$ is an upper bound for $B$}.
	\end{align*}
\end{exercise}

\begin{exercise}
	\begin{enumerate}[label=(\alph*)]
		\item \begin{align*}
		&s = \sup(c + A) \\
		\lra &s \text{ is the least upper bound for } c + A \\
		\lra &s - c \text{ is the least upper bound for } A \\
		\lra &s - c = \sup A \\
		&s = c + \sup A
	\end{align*}
		\item \begin{align*}
			&s = \sup(cA) \\
			&\lra s \text{ is the least upper bound for } cA \\
			&\lra \frac{s}{c} \text{ is the least upper bound for } A \\
			&\lra \frac{s}{c} = \sup A \\
			&s = c \sup A
		\end{align*}
		\item If $c < 0$, $\sup(cA) = -c\sup(A)$.
	\end{enumerate}
\end{exercise}

\begin{exercise}
\begin{enumerate}[label=(\alph*)]
	\item $\sup, \sqrt{10}; \inf, 1$
	\item $\sup, 1; \inf, 0$
	\item $\sup, \frac{1}{2}; \inf, \frac{1}{3}$
	\item $\sup, \infty; \inf, -\infty$
\end{enumerate}
\end{exercise}

\begin{exercise}
	If $a \geq a', \forall a' \in A$, and $a \in A$, then
	\begin{equation}
		\forall \epsilon > 0, a - \epsilon < a,
	\end{equation}
	so $a$ is the least upper bound for $A$, and $a = \sup A$.
\end{exercise}

\begin{exercise}
	Let \begin{equation}
	\epsilon = \sup B - \sup A > 0.
	\end{equation}
	since $s_b = \sup B$, $\exists b \in B \mid b > s_b - \epsilon / 2$. Since $s_b - \frac{\epsilon}{2} > \sup A$, then $b \geq \sup A$, so this $b \in B$ is an upper bound for $A$.
\end{exercise}

\begin{exercise}
\begin{enumerate}[label=(\alph*)]
	\item True (take the largest element)
	\item False $\sup (0, 2) = 2$, but $2 > a \in (0, 2)$, but $\sup A = 2 \not < 2 = L$.
	\item False $A = (0, 2), B = [2, 3)$. We have that $\sup A = \inf B$
	\item True
	\item False (take $A = B = (0, 2)$)
\end{enumerate}
\end{exercise}

%%% 1.4 %%%
\setcounter{subsection}{4}
\setcounter{exercise}{0}

\begin{exercise}
	If $a<0$, then we have two cases,
	\begin{enumerate}
		\item If $b > 0$, then $a < 0 < b$.
		\item If $b = 0$, then we can take $-b, -a$, which satisfies $0 \leq -b < -a$, and apply Theorem 1.4.3.
	\end{enumerate}
\end{exercise}

\begin{exercise}
\begin{enumerate}[label=(\alph*)]
	\item If $a, b \in \mathbb{Q}$, then 
	\begin{align*}
		a &= \frac{a_1}{a_2} \\
		b &= \frac{b_1}{b_2} \\
		\lra a + b &= \frac{a_1b_2 + a_2b_1}{a_2b_2} \in \mathbb{Q}
	\end{align*}
	\item AFSOC $at \in \mathbb{Q}$. But if $a = \frac{a_1}{a_2}$, this implies
	\begin{equation*}
		t = \frac{a_2}{a_1}\frac{m}{n} \in \mathbb{Q},
	\end{equation*}
	which is a contradiction, so we must have that $at \in\mathbb{I}$
	\item $\mathbb{I}$ is not closed under addition or multiplication.
\end{enumerate}
\end{exercise}

\begin{exercise}
	We can apply Theorem 1.4.3, to find $a < q < b, q \in \mathbb{Q}$, and then subtract an irrational number such as $\sqrt{2}$ to end up at 
	\begin{equation}
		a - \sqrt{2} < q - \sqrt{2} < b - \sqrt{2},
	\end{equation}
	where $q- \sqrt{2} \in \mathbb{I}$.
\end{exercise}

\begin{exercise}
	Suppose $\exists b$ lower bound such that $b > 0$. Then by Archimedean Property of $\mathbb{R}$, $\exists n \in \mathbb{N}$ such that $\frac{1}{n} < b$, which means $b$ is not a valid lower bound. Thus $b \leq 0$, and 0 is a valid lower bound so the $\inf$ is 0.
\end{exercise}

\begin{exercise}
AFSOC $\exists \alpha \in \bigcap_{n=1}^\infty (0, \frac{1}{n})$. Then $\alpha >0$, but by Archimedean property of reals, we have that $\exists n \in \mathbb{N} \mid \frac{1}{n} < \alpha$. Since $\alpha \not\in (0, \frac{1}{n}$, $\alpha \not\in \bigcap_{n=1}^\infty (0, \frac{1}{n})$, a contradiction. Thus we conclude the set is empty.
\end{exercise}

\begin{exercise}
\begin{enumerate}[label=(\alph*)]
	\item If $\alpha^2 > 2$, then 
	\begin{align*}
		\pa{a - \frac{1}{n}}^2 &= \alpha^2 - \frac{2\alpha}{n} + \frac{1}{n^2} \\
		&> \alpha^2 - \frac{2\alpha}{n}
	\end{align*}
	choose $\frac{1}{n_0} < \frac{\alpha^2 - 2}{2\alpha}$. Then
	\begin{align*}
		\pa{a - \frac{1}{n_0}}^2 &> \alpha^2 - \frac{2\alpha}{2\alpha}(\alpha^2 - 2)\\
		&> 2
	\end{align*}
	but $\alpha - \frac{1}{n_0} < \alpha$, so $\alpha$ is not the least upper bound for the set.
	\item Just replace $\sqrt{2}$ with $\sqrt{b}$
\end{enumerate}
\end{exercise}

\begin{exercise}
	Take the minimum from the set. Assign as $i$, then remove the minimum from the set. Repeat for $i+1$ and so on.
\end{exercise}

\begin{exercise}
\begin{enumerate}[label=(\alph*)]
	\item If both are finite, then their union is finite and trivially countable. If one is finite, then first enumerate elements of the finite set. Then map the rest of $\mathbb{N}$ to the countably infinite set. If both are countably infinite, map one set to odds and the other to evens.
	\item Induction only holds for finite integers, not infinity.
	\item We can arrange each $A_n$ into row $n$ of a $\mathbb{N}\times \mathbb{N}$ matrix. Then, we enumerate by diagonalization.
\end{enumerate}
\end{exercise}

\begin{exercise}
\begin{enumerate}[label=(\alph*)]
	\item If $A \tat B$, then there is a 1-to-1 mapping. We can just take the inverse of the mapping to derive $B \tat A$.
	\item If we have $f: A \rightarrow B$, $g: B \rightarrow C$, then we can compose the functions so $g(f(x)): A \rightarrow C$.
\end{enumerate}
\end{exercise}

\begin{exercise}
The set of all finite subsets of $\mathbb{N}$ can be ordered in increasing order by the sum of each subset.
\end{exercise}

\begin{exercise}
\begin{enumerate}[label=(\alph*)]
	\item $f(x) = x$
	\item Interweave the decimal expansion of $x, y$, e.g. 
	\begin{equation}
	f(x, y) = 0.x_1y_1x_2y_2x_3y_3\dots
	\end{equation}
\end{enumerate}
\end{exercise}

\bx{
\ea{
	\item \begin{align*}
		\sqrt{2}\,:\, x^2 -2 2 = 0 \\
		\sqrt[3]{2}\,:\,x^3 - 2 = 0
	\end{align*}
	$\sqrt{3} + \sqrt{2}$ is not as trivial, so we will do it out in more steps.
	
	There are two approaches to finding the integer coefficient polynomial. One is to take advantage of symmetry, and derive that 
	\begin{equation}
	\prod (x - (\pm \sqrt{3} \pm \sqrt{2})
	\label{eq:sym}
	\end{equation}
	will work (using loose notation of course). A more general technique is to notice that 
	\begin{align*}
		x &= \sqrt{3} + \sqrt{2} \\
		x^2 &= 5 + 2\sqrt{6} \\
		(x^2 - 5)^2 &= 24 \\
		x^4 - 10x^2 + 1 &= 0.
	\end{align*}
	Notice that this is actually the exact same answer we get in (\ref{eq:sym}) if you work it out.
	\item Each $\abs{A_n} = \abs{\mathbb{N}^n}$, which is countable
	\item We proved earlier in Theorem 1.4.13 that a countably infinite union of countable sets is countable. Since there are a countable number of algebraic numbers, and reals are uncountable, we conclude that transcendentals are also uncountable.
}
}

\bx{
\ea{
	\item INCOMPLETE
}
}

%%% 1.5 %%%
\setcounter{subsection}{5}
\setcounter{exercise}{1}

\bx{
\ea{
	\item Because $b_1$ differs from $f(1)$ in position 1
	\item $b_i$ differs from $f(i)$ in position $i$.
	\item We reach a contradiction that we can enumerate the elements of $(0, 1)$, and thus $(0, 1)$ is uncountable.
}
}

\bx{
\ea{
	\item $\frac{\sqrt{2}}{2} \in (0, 1)$ but is irrational
	\item We can just define our decimal representations to never have an infinite string of 9s
}
}

\bx{
Suppose $S$ is countable. Then we can enumerate the elements of $S$ using the natural numbers. Now, consider some $s = (s_1, s_2, \dots)$, where
\begin{equation}
	s_i = \begin{cases}
		0, \text{if } f(i), \text{position } i = 1
		1, \text{otherwise}
	\end{cases}
\end{equation}
Then since $s \neq f(i) \forall i$, $s \not \in S$. But this is a contradiction since $s$ only contains elements 0 or 1, and thus should be in $S$. Thus, we conclude that $S$ is uncountable.
}

\bx{
\ea{
	\item \begin{equation}
	\mathcal{P}(A) = \{\emptyset, \{a\}, \{b\}, \{c\}, \{a,b\}, \{a,c\}, \{b,c\}, \{a,b,c\}\}
	\end{equation}
	\item Each element has two choices when constructing a subset of $A$. To be, or not to be\footnote{sorry, had to do it}, in the set.
}
}

\bx{
\ea{
	\item Many different answers.
	\begin{align*}
		&\{(a, \{a\}), (b, \{b\}), (c, \{c\})\} \\
		&\{(a, \emptyset), (b, \{b\}), (c, \{c\})\} \\
	\end{align*}
	\item \begin{equation*}
	\{(1, \{1\}), (2, \{2\}), (3, \{3\}), (4, \{4\})\}.
	\end{equation*}
	\item Because $\abs{\mathcal{P}(A)} > \abs{A}$ and same for $B$.
}
}

\bx{
	\begin{enumerate}
		\item $B = \emptyset$
		\item $B = \{a, d\}$
	\end{enumerate}
}

\bx{
\ea{
	\item AFSOC $a' \in B$. Then that means $a \not \in f(a')$. But this is a contradiction since $a' \in B = f(a')$.
	\item AFSOC $a' \not\in B = f(a')$. Then since $a' \not\in f(a')$, $a' \in B$, but that is a contradiction.
}
}

\bx{
\ea{
	\item This is the same as $\mathbb{N} \times \mathbb{N}$, which  is countable.
	\item Uncountable, since $\mathcal{P}(\mathbb{N})$ is uncountable.
	\item Is this question asking for the number of antichains or if there is an antichain with uncountable cardinality? The latter is obvious, and \textit{no} is the answer since any subset of $\mathbb{N}$ is countable. The first case probably uncountable???
}
}
