%% 4.5 %%

\section{The Intermediate Value Theorem}
\setcounter{exercise}{0}

\bx{
If we have a continuous function $f$ on $[a, b]$,
and we know that $[a, b]$ is connected,
then that means $f([a, b])$ is also connected.

By the property of connected sets, it follows directly that if we have 
$f(a) < L < f(b)$, we can always find this $L \in f([a, b])$.
Then, this $L$ must correspond to some $c \in (a, b)$, which proves
the IVT.
}

\bx{
\ea{
\item No, you can imagine taking $(-1, 1)$ to $[1, 2)$ with $f(x) = x^2+1$.
The idea for making this was that the bounds don't necessarily have to come
from the open endpoints.
\label{chap4:open_to_closed}
\item Can use the same example from part \ref{chap4:open_to_closed}
\item Yes. If we have bounded closed interval, then it is compact, so a continuous 
function will take it to another compact interval, which is bounded and closed.
This function will also preserve connectedness.
}
}

\bx{
No, since $\mathbb{R}$ is connected, and if $f$ is continuous,
it will preserve connectedness. $\mathbb{Q}$ is not connected.
}

\bx{
Suppose we have some function $f$ with the Intermediate value property 
and is also increasing. We want to show that $f$ is continuous.

Suppose we have some $\epsilon > 0$ and a point $c \in [a, b]$.
Let 
\begin{equation*}
L_1 = \max(f(a), f(c) - \epsilon), \min(f(b), L_2 = f(c) + \epsilon)
\end{equation*}
By the IVT property, we know $\exists x_1, x_2 \in [a, b]$
where $f(x_1) = L_1, f(x_2) = L_2$.

By the increasing property of $f$, we have that any $x \in [x_1, x_2]$ will satisfy
\begin{equation*}
  \abs{f(x) - f(c)} < \epsilon
\end{equation*}
since $f(x_1) \leq f(c) \leq f(x_2)$.

Choose $\delta = \min(\abs{c - x_1}, \abs{c - x_2})$, and this will work for any $\epsilon$
challenge.
}

\bx{
To complete the IVT proof with the Axiom of Completeness, we are trying to show 
$\exists c \in (a, b)$ such that $f(c) = 0$ if $f(a) < 0 < f(b)$, and $f$ is continuous.
With our definition of $K = \pbrac{x \in [a, b] : f(x) \leq 0}$,
since we know $K$ is bounded above by $b$, and $a \in K$, $K$ is not empty.
By the Axiom of Completeness, we know $c = \sup K$ exists.

\begin{itemize}
  \item We cannot have  $f(c) > 0$ since all elements of $K$ satisfy $f(x) \leq 0$.
  \item We cannot have $f(c) < 0$, because it is not an upper bound that includes $0$.
  \item Therefore, we conclude $f(c) = 0$, which means we have found a $c \in (a, b)$ such that $f(a) < f(c) = 0 < f(b)$
\end{itemize}

To extend it for a general $f(a) < c < f(b)$, just consider the function $f'(x) = f(x) - c$,
and then we can do the $f(a) < 0 < f(b)$ case, which we have already proved.
}

\bx{
With the binary set construction described in the text, once we take the 
$\bigcap_{n=1}^\infty I_n$, we must have an element in this intersection by NIP.
Call this element $x$.

Now, our claim is that $f(x) = 0$.
Notice that $a_n$ and $b_n$ converge to $x$, 
since $\abs{a_n - b_n} < \frac{\abs{a-b}}{2^n}$.
Then, since $f$ is continuous, $f(a_n)$ and $f(b_n)$ must converge to the same $L$.
Since $f(a_n) < 0, f(b_n) \geq 0$, it must be the case that $\lim_{n\to\infty} f(a_n) = \lim_{n\to\infty} f(b_n) = 0$,
since otherwise, for example if $L > 0$, then $\abs{f(a_n) - L} \geq \epsilon_0$.
Therefore, we have found $x \in (a, b)$ such that $f(x) = 0$.
\label{chap4:IVT_NIP}
}

\bx{
Construct the folowing sequence of intervals, and let $c_{n-1} = \frac{a_{n-1} + b_{n-1}}{2}$,
\begin{equation*}
  I_n = [a_n, b_n] = \begin{cases}
    [0, 1], &n=0\\
    [c_{n-1}, b_{n-1}], &\text{if } f(c_{n-1}) > c_{n-1}\\
    [a_{n-1}, c_{n-1}], &\text{if } f(c_{n-1}) \leq c_{n-1} 
  \end{cases}
\end{equation*}
Take $x \in \bigcap_{n=1}^\infty I_n$. By the NIP this $x$ exists.

Now, our claim is that $f(x) = x$. This is similar to the proof in Exercise \ref{chap4:IVT_NIP},
where we use the continuity of $f$ and that $(a_n) \rightarrow x, (b_n) \rightarrow x$,
so that $\lim_{n\to\infty} f(a_n) = a_n$, and therefore we conclude $f(x) = x$.

A more straightforward proof with IVT is to define $g(x) = f(x) - x$.
Then we have $g(0) = f(0) - 0 = f(0) \in [0, 1] \Rightarrow g(0) \geq 0$, 
and also $g(1) = f(1) - 1 \Rightarrow f(1) - 1 \in [-1, 0] \Rightarrow g(1) \leq 0$.
Putting these two together, we have either $g(0) = 0, g(1) = 0$, or neither are equal to 1.
If either are equal to one, then we have the trivial case where $g(c) = 0 \Rightarrow f(c) = c$
for $c = 0, 1$. 

Otherwise, if $g(0) \neq 0, g(1) \neq 0$,
then we have $g(1) < L < g(0)$,
and then we can use the IVT to say $\exists c$
such that $g(1) < g(c) < g(0)$ and $c \in (0, 1)$.
This implies that for this $c$, that $f(c) = c$.

Therefore, we have shown in all cases that we have a fixed point in this range.
\label{chap4:ex_fixed_point}
}

\bx{
We cannot, since there are times that are ambiguous. 

For example, consider the set of times from 1:00 - 2:00.
Here, the hour hand goes from $30^\circ - 60^\circ$,
while the minute hand goes from $0^\circ - 360^\circ$,

We can find a time that is isomorphic to this time range by considering that
$30^\circ - 60^\circ$ is the 5-10 minute range, and if the minute hand is 
past some hour enough to the point where it crosses the same point where 
the hour point would be with a minute in the range of 5-10 minutes,
then we would have an isomorphic time.

For example, consider when the minute hand goes past 2. The minute hand will 
go from 2 - 3, but in between there, at some point it can represent some 2-hour time 
where the minute is between 5-10 minutes.

There's probably a better setup and application of IVT, but this is enough to 
convince me that there are isomorphic times on the clock.
}