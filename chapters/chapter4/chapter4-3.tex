%% 4.3 %%

\section{Combinations of Continuous Functions}
\setcounter{exercise}{0}

\bx{
\ea{
\item For some $\epsilon > 0$, choose $\delta = \epsilon^3$, then 
\begin{equation*}
  \abs{
    f(x) - 0
  } = \abs{\sqrt[3]{\epsilon^3}} < \epsilon
\end{equation*}

\item If we are given some $\epsilon > 0$, $c \in \mathbb{R}$,
choose $\delta = \epsilon\sqrt[3]{c^2}$, then
\begin{align*}
  \abs{
    \pa{
      \sqrt[3]{x} - \sqrt[3]{c}
    }
    \pa{
      \sqrt[3]{x^2} + \sqrt[3]{xc} + \sqrt[3]{c^2}
    }
  }
  &= \abs{x - c}\\
  \abs{
    \sqrt[3]{x} - \sqrt[3]{c}
  } 
  &= 
  \frac{
    \abs{x - c}
  }{
    \sqrt[3]{x^2} + \sqrt[3]{xc} + \sqrt[3]{c^2}
  }\\
  &<
  \frac{
    \abs{x - c}
  }{
    \sqrt[3]{c^2}
  }\\
  &= \frac{
    \epsilon\sqrt[3]{c^2}
  }{
    \sqrt[3]{c^2}
  }
  = \epsilon
\end{align*}
}
}

\bx{
\ea{
\item For any $\epsilon > 0$, 
Since $g$ is continuous at $f(c)$, choose $\delta_g$
such that if $f(x) \in V_{\delta_g}(f(c))$,
then 
\begin{equation*}
  \abs{g(f(x)) - f(c)} < \epsilon.
\end{equation*}

Now, since $f(x)$ is continuous at $c$, there is a $\delta_f$ such that 
if $x \in V_{\delta_f}(c)$, then
\begin{equation*}
  \abs{f(x) - f(c)} < \delta_g.
\end{equation*}

Choose this $\delta_f$, then we will have for any $\epsilon > 0$,
\begin{equation*}
  x \in V_{\delta_f}(c) \Rightarrow \abs{g(f(x)) - f(c)} < \epsilon
\end{equation*}

\item Using sequential characterization, since $f(x)$ is continuous at $c$,
we know that
$(x_n) \rightarrow c$ implies $f(x_n) \rightarrow f(c)$.

Now, $g \circ f(x)$ is well-defined on $A$, so if we 
let $y_n = f(x_n)$,
we know that since $g(y)$ is continuous at $f(c)$, 
$y_n \rightarrow f(c)$ means $g(y) \rightarrow g(f(c))$.

Putting these together, we just showed $(x_n) \rightarrow c$ implies $f(x_n) \rightarrow f(c)$,
which finally implies $g(f(x_n)) \rightarrow g(f(c))$.
By the sequential characterization, we can conclude that $g \circ f(x)$ is continuous at $x=c$.
}
}

\bx{
For any $\epsilon > 0$, choose $\delta = \frac{\epsilon}{a}$.
Then 
\begin{equation*}
  \abs{f(x) - f(c)} = \abs{a(x - c)} < a \frac{\epsilon}{a} < \epsilon
\end{equation*}
}

\bx{
\ea{
\item For $\epsilon > 0$, choose $\delta < 1$, then $x, c \in \mathbb{Z}$ and $0 < \abs{x - c} < \delta$
implies that $x = c$, so $\abs{f(x) - f(c)} = 0 < \epsilon$.

\item For any isolated point, we can choose a $\delta$-neighborhood small enough so that $0 < \abs{x - c} < \delta$ 
implies $x = c$, since there are no other elements in the domain near $c$ other than $c$ itself. 
This will make $\abs{f(x) - f(c)} = 0 < \epsilon$.
}
}

\bx{
We can find a $\delta_g$ so that $x \in V_{\delta_g}(c)$ means $\abs{g(x) - g(c)} < \epsilon = \abs{g(c)}$.
This means $\forall x, g(x) \neq 0$, since it is too far away from 0, so therefore the denominator is always 
nonzero and $f(x)/g(x)$ is defined in this open interval.
}

\bx{
\ea{
\item Choose some $c \in \mathbb{R}$, we have 2 cases,
\begin{enumerate}[label=\arabic*.]
  \item $f(c)$ is rational.
  We can choose an irrational sequence $(x_n) \rightarrow c$,
  so that $f(x_n) \rightarrow 0$, since $\forall x_n, f(x_n) = 0$, since $x_n$ is irrational.
  However, we know that $f(c) = 1$, since $c$ is rational.

  \item $f(c)$ is irrational.
  We can choose an rational sequence $(x_n) \rightarrow c$,
  so that $f(x_n) \rightarrow 1$, since $\forall x_n, f(x_n) = 1$, since $x_n$ is rational.
  However, we know that $f(c) = 0$, since $c$ is irrational.
\end{enumerate}
Therefore, we conclude the Dirichlet function is everywhere-discontinuous.
\label{chap4:dirichlet_discontinuous}

\item Proving Thomae's function is discontinuous at every rational point is 
the same proof as showing Dirichlet's function is discontinuous at every rational point.
See part (
  \ref{chap4:dirichlet_discontinuous}
).

\item I'm extremely confused about this hint $t(x) \geq \epsilon$. 
All of these $x$ that satisfy this must be rational numbers, and they must be isolated
if we force a limit on their denominator, which is what $t(x) \geq \epsilon$ does.
I think then the argument goes, take the complement of this set, so $\pbrac{x : t(x) < \epsilon}$,
then we know this is bounded somehow? And then we choose a $\delta$ so that 
our neighborhood only contains elements from this set? Maybe something about
an open set existing in this set too, and we can choose that. Overall...unsure.
\TODO

Using Theorem 4.3.2 (iii), which is the $\epsilon, \delta$-neighborhood argument,
What you can do is choose a $\delta$ so small so that to represent $\delta$ as a rational,
you must use a denominator large enough such that $\frac{m}{n}$, $\frac{1}{n} < \epsilon$.
}
}

\bx{
To show $K$ is a closed set, we need to show that $K$ contains all of its limit points.
Take any $(x_n) \rightarrow l$ from $K$. Since we know $h(x)$ is continuous, any 
$(x_n) \rightarrow l$ implies $h(x_n) \rightarrow h(l)$.
Now, since all terms of $h(x_n)$ are 0, we must have $h(l) = 0$, 
otherwise, if $h(1) = y$, then choose $\epsilon < \abs{y}$, and you cannot prove 
convergence.
We conclude that $l \in K$, and 
therefore $K$ contains all of its limit points and is closed.
}

\bx{
\ea{
\item \AFSOC $\exists c \in \mathbb{R}, f(c) \neq 0$. 
Then $\exists (x_n) \rightarrow c, x_n \in \mathbb{Q}$.
Since $f$ is continuous on $\mathbb{R}$, we must have $(x_n) \rightarrow c$
implies $f(x) \rightarrow f(c)$. But $f(x_n) = 0$, and $f(c) \neq 0$, so
this means $f$ is not continuous at $c$. But this is a contradiction, since 
$f$ is continuous on $\mathbb{R}$.

\item There is no claim about continuity, so no. We can have $f(r) = g(r), r \in \mathbb{Q}$,
but have $f(i) = -1, g(i) = 1, i \in \mathbb{I}$. Then $f$ and $g$ are not the same function.
}
}

\bx{
\ea{
\item We can do the $\epsilon$-$\delta$ proof, 
but choose $\delta = \frac{\epsilon}{c}$, so for any $z \in \mathbb{R}$, 
we can show that for $0 < \abs{x - z} < \delta$, that $\abs{f(x) - f(z)} \leq c\abs{x - z} < c \frac{\epsilon}{c} = \epsilon$.

\item We notice that for every $y_{n+1}$, its distance from $y_n$ is $c$ times the 
distance of $\abs{y_n - y_{n-1}}$. Since $c \in \pa{0, 1}$, this distance will get smaller every single time.
Namely, 
\begin{equation*}
  \abs{y_{n+1} - y_{n}} = \abs{f(y_n) - f(y_{n-1})} \leq c \abs{y_n - y_{n-1}} \leq c^{n-1}\abs{y_2 - y_1}
\end{equation*}
So for any $\epsilon > 0$, we can choose $N$ such that $c^{N-1} < \frac{\epsilon}{\abs{y_2-y_1}}$,
and then for subsequent $n \geq N$, we have $\abs{y_n - y} < \epsilon$. Since $\lim y_n = y$, this sequence converges to $y$,
which is equivalent to showing it is a Cauchy sequence.
\label{chap4:part_y_comp}

\item We can choose the sequence $x_n = y_n$ where $y_n$ was defined in 
part (
  \ref{chap4:part_y_comp}
), and we see that $(x_n) \rightarrow y$, and since $f$ is continuous on $\mathbb{R}$,
we conclude that $f(x_n) \rightarrow f(y)$. Now, $f(x_n) = x_n+1$, so $f(x_n) \rightarrow y$.
This means $f(y) = y$. To prove uniqueness, we need to show no other $x$ exists such that $f(x) = x$.
\AFSOC such an $x$ exists, then we can do the following:
\begin{itemize}
  \item We have $\abs{f(x) - f(y)} \leq c \abs{x - y}$.
  \item If we apply this property again with $f(f(x)), f(f(y))$, we get 
  \begin{align*}
    \abs{f(f(x)) - f(f(y))} &\leq c\abs{f(x) - f(y)}\\
    \abs{f(x) - f(y)} &\leq c^2 \abs{x - y} \tag{Using $f(y) = y, f(x) = x$ property, and the prev result}
  \end{align*}
  \item We can continue this indefinitely, and show that $y = f(y) = f(x) = x$, which means $y$ is unique.
\end{itemize}

\item Let us say $(x_n) \rightarrow L$. Then we can find for any $\epsilon > 0$,
a $N$ such that for $n \geq N$, $\abs{x_n - L} < \epsilon$.

Now, if we take $\abs{f(x_n) - f(L)}  = \abs{x_n - f(L)} \leq c\abs{x_n - L} < \epsilon$, 
we see this sequence also converges to $f(L)$, which means $f(L) = L$. In the previous part, we showed 
$y$ is the unique fixed point, so we conclude $L=y$, and thus sequence also converges to $y$.
}
}

\bx{
\ea{
\item We have two problems,
\begin{itemize}
  \item $f(0 + 0) = f(0) = f(0) + f(0)$. If we have $x = 2x$, $x=0$ is the
  only solution, so $f(0) = 0$.
  \item $f(x - x) = f(0) = f(x) + f(-x) \Rightarrow -f(x) = f(-x)$.
\end{itemize}

\item For any $\epsilon > 0$, since $f(0)$ is continuous, 
$\exists \delta$ such that $\abs{f(\delta) - f(0)} < \epsilon$.
Now, if we have $c \in \mathbb{R}$, if we have some 
$(x_n) \rightarrow c$, we can choose $N$ such that for $n \geq N$,
$\abs{x_n - c} < \delta$. Then,
\begin{equation*}
  \abs{f(x_n) - f(c)} = \abs{f(x_n - c)} = \abs{f(\delta)} < \epsilon
\end{equation*}

\item Just expand $f(1 + 1 + \cdots + 1)$, so e.g. $f(3) = f(1 + 1 + 1) = f(1) + f(1 + 1) = f(1) + f(1) + f(1) = 3f(1) = 3k$.
Using $f(-x) = f(x)$, we can show $f(z) = kz, z \in \mathbb{Z}$. 
For rationals, let's say we have $\frac{m}{n} \in \mathbb{Q}$, where $m, n \in \mathbb{Z}$, then we can write 
$f(m)$ as $f(n \cdot \frac{m}{n})$, which means $f(m) = nf\pa{\frac{m}{n}}$. Now, since $m \in \mathbb{Z}$,
we know $f(m) = mk$, so $mk = n f\pa{\frac{m}{n}} \Rightarrow f\pa{\frac{m}{n}} = \frac{m}{n}k$.

\item We only haven't shown the property $f(x) = kx$ is true for irrationals.
We can do this by constructing a rational sequence $(x_n) \rightarrow c$ for any $c \in \mathbb{R}$.
Now since $f$ is continuous on $\mathbb{R}$, we conclude that $f(x_n) \rightarrow f(c)$.
We also know for every $f(x_n) = kx_n$, so $f(c) = kc$. 
}
}

\bx{
\ea{
\item Let 
\begin{equation*}
  f(x) = 
  \begin{cases}
    x, &x \not\in \mathbb{Z}\\
    x+1, &x \in \mathbb{Z}
  \end{cases}
\end{equation*}

\item We basically define a function that is ``normal'', except 
inside $(0, 1)$ we make it behave crazy like the Dirichlet function.
We make sure that the surrounding area connects to this region smoothly,
so that $0, 1$ are continuous.
\begin{equation*}
  f(x) = 
  \begin{cases}
    0, x \in \mathbb{Q} \cap (0, 1)\\
    1, x \in \mathbb{I} \cap (0, 1)\\
    x, x \not\in \mathbb{Z}
  \end{cases}
\end{equation*}
\label{chap4:discontinuous_in_0_1}

\item Basically the same as part (
  \ref{chap4:discontinuous_in_0_1}
) except we make the surrounding area of the interval disconnected,
so that $0, 1$ are discontinuous.
\begin{equation*}
  f(x) = 
  \begin{cases}
    0, x \in \mathbb{Q} \cap [0, 1]\\
    1, x \in \mathbb{I} \cap [0, 1]\\
    x+100, x \not\in \mathbb{Z}
  \end{cases}
\end{equation*}

\item We have to pay attention to $0$, since there exists a subsequence 
$\in A$ that converges to 0. Since this subsequence will converge to 0, 
we just have to make sure that $f(x_n) \rightarrow f(0)$.
\begin{equation*}
  f(x) = 
  \begin{cases}
    x, &x \in A\\
    0, &x \not\in A
  \end{cases}
\end{equation*}
}
}

\bx{
\ea{
\item We can construct two subsequences that converge to $c$, one which is 
always in $C$, and the other that is not. Then $g(x_n)$ will converge to 1 for the first subsequence,
but 0 for the second. Therefore every $c \in C$ is discontinuous.

\item If $c \not\in C$, then $c$ is part of an open set, which means there exists an $\epsilon$
neighborhood around it, where all points $x$ are such that $g(x) = 0$.
}
}