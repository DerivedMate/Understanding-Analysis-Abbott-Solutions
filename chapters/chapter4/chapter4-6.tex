%% 4.6 %%

\section{Sets of Discontinuity}
\setcounter{exercise}{0}

\bx{
\ea{
\item 
\begin{equation*}
  f(x) = 
  \begin{cases}
    1, &x \in \mathbb{Z}\\
    0, &\text{otherwise}
  \end{cases}
\end{equation*}
\item 
\begin{equation*}
  f(x) = 
  \begin{cases}
    \text{Dirichlet's Function}, &x \in (0, 1]\\
    100, &\text{otherwise}
  \end{cases}
\end{equation*}
}
}

\bx{
\textbf{(Left-hand Limit)}
Given a limit point of a set $A$ and a function $f:A \rightarrow \mathbb{R}$,
we write 
\begin{equation*}
  \lim_{x\to c^-} f(x) = L
\end{equation*}
if $\forall \epsilon>0$ $\exists \delta > 0$ such that 
\begin{equation*}
  \abs{f(x) - L} < \epsilon \text{ whenever } 0 < c - x < \delta
\end{equation*}
Equivalently, $\lim_{x\to c^-} f(x) = L$ if $\lim f(x_n) = L$
for all sequences $(x_n)$ satisfying $x_n < c$ and $\lim(x_n) = c$.
}

\bx{
($\Rightarrow$)
If $\lim_{x\to c} f(x) = L$, then all sequences that converge to $c$
will have $\lim f(x_n) = L$, so we have 
$\lim_{x\to c^+} f(x) = L$ and
$\lim_{x\to c^-} f(x) = L$.

($\Leftarrow$)
If we have 
$\lim_{x\to c^+} f(x) = L$ and
$\lim_{x\to c^-} f(x) = L$,
we want to be able to show for any sequence $(x_n) \rightarrow c$,
that we also have $\lim f(x_n) = L$.
For any $(x_n)$, we know there must be an infinite number of elements 
greater than $c$ and/or less than $c$ if $(x_n) \to c$, 
so WLOG take this subsequence that is all greater than $c$, which we know satisfies 
$\lim f(x_n) = L$.
}

\bx{
($x \to c^-$)
For any $\epsilon > 0$, we can find $f(x') = f(c) - \epsilon$
by the IVT, and since $f$ is monotone, we know $0 < c - x < \delta = \abs{c - x'}$
will satisfy the $\epsilon$ challenge, since any $x$ in this range 
will satisfy $f(x') \leq f(x) \leq f(c)$, since $x' \leq x \leq c$,
and $\abs{f(x) - f(x')} < \epsilon$.
Similar logic applies to the other direction.

Since the limit exists at every point, but could potentially be different, a monotone function 
can only have jump discontinuity.
}

\bx{
For any monotone function $f$ with a jump discontinuity at $c$, where 
$\lim_{x \to c^-} f(x) = L^-, \lim_{x\to c^+} f(x) = L^+$,
it must be the case that $L^- < L^+$. In addition, because $\mathbb{Q}$ is 
dense, we know that $\exists r \in \mathbb{Q}$ such that $L^- < r < L^+$.
Now, there is a slight concern that maybe this $r$ could be repeated at other discontinuities,
but since we know $f$ is monotone, we know that for $c_1 < c_2$,
we must have $L_1^- < L_2^-$ so therefore $r_1 < r_2$.
Therefore, we can map all of $D_f$ to unique $r \in \mathbb{Q}$,
which means $D_f$ is a subset of $\mathbb{Q}$. Since $\mathbb{Q}$ 
is countable, we conclude $D_f$ is finite or countable.
}

\bx{
\ea{
\item $\mathbb{R}$ is closed
\item $\mathbb{R} \setminus \pbrac{0} = \bigcup_{n=1}^\infty \left(-\infty, -\frac{1}{n}\right] \cup \left[\frac{1}{n}, \infty\right)$
\item $\mathbb{Q} = \bigcup_{r \in \mathbb{Q}} \pbra{r}$, $\mathbb{Q}$ is countable
\item $\mathbb{Z} = \bigcup_{n \in \mathbb{Z}} \pbra{z}$, $\mathbb{Z}$ is countable
\item $\left(0, 1\right] = \bigcup_{n=1}^\infty \pbra{\frac{1}{n}, 1}$
}
}

\bx{
The set that $f$ is $\alpha$ continuous is some aribitrary union of 
open sets $\pa{x - \delta, x + \delta}$, which means $\overline{D_\alpha}$
is open. Then, we conclude $D_\alpha$ is closed, since it is the complement of 
an open set.
}

\bx{
If we have $\alpha_1 < \alpha_2$, we know that the set of
$\alpha_1$-continuous points is a subset of the $\alpha_2$-continuous 
points since if $\exists \delta$ such that $y, z \in \pa{x - \delta, x+\delta}$
and $\abs{f(y)-f(z)} < \alpha_1 < \alpha_2$.
Therefore, $\overline{D_{\alpha_1}} \subseteq \overline{D_{\alpha_2}} \Rightarrow D_{\alpha_2} \subseteq D_{\alpha_1}$.
}

\bx{
If $f$ is continuous at $x$, we know for any $V_\epsilon(f(c))$, we can find 
$\exists \delta, x \in V_\delta(c)$ such that $f(x) \in V_\epsilon(f(c))$.
Choose $\epsilon = \alpha/2$, then we know that we have a delta that satisfies 
for $y, z \in V_\delta(x)$,
\begin{align*}
  \abs{f(y)-f(z)} 
  &= \abs{f(y) - f(c) - \pbra{f(z) - f(c)}}\\
  &\leq \abs{f(y) - f(c)} + \abs{f(z) - f(c)}\\
  &< \alpha/2 + \alpha/2 = \alpha
\end{align*}

Since $f$ is $\alpha$-continuous at every point it is continuous,
we conclude that for points where it is not $\alpha$-continuous,
it must be not in the set of continuities of $f$, which is namely $D_f$.
Therefore, $D_\alpha \subseteq D_f$.
}

\bx{
If $f$ is not continuous at $x$, it must be the case that $f$ is not $\alpha$-continuous
at $x$ for some $\alpha > 0$, otherwise, we can show that $f$ 
is continuous at this point by showing that 
$\exists \delta, \forall y \in V_\delta(x)$, we have $\abs{f(y) - f(x)} < \alpha$.

To show that 
\begin{equation*}
  D_f = \bigcup_{n=1}^\infty D_{
    \frac{1}{n}
  },
\end{equation*}
for any $\alpha > 0$, $\exists m \in \mathbb{N}$ such that $0 < \frac{1}{m} < \alpha$.
We showed $\frac{1}{m} < \alpha$ implies $D_\alpha \subseteq D_{\frac{1}{m}}$, 
so we can conclude for any discontinuity of $f$, it will be included in 
some $D_{\frac{1}{n}}$.

This union of a countable number of closed sets shows that the discontinuities
of $f$ is a $F_\sigma$ set.
}