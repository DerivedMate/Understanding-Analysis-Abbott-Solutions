\appendix
\chapter{Extras}
\section{Useful Tools}

Collection of useful tools and methods to solve problems.

\begin{tip}
  Template for a proof that $(x_n) \rightarrow x$:
  \begin{itemize}
    \item Let $\epsilon > 0$ be arbitrary 
    \item Demonstrate a choice for $N \in \mathbb{N}$. 
    This step usually requires the most work, almost all of
    which is done prior to actually writing the formal proof.
    \item Now, show that $N$ works.
    \item Assume $n \geq N$
    \item With $N$ well chosen,  you should be able to show $\abs{x_n - x} < \epsilon$.
  \end{itemize}
\end{tip}

\section{Cool Things}
\begin{itemize}
  \item In Chapter 2, we learn that addition in infinite sums is not commutative.
  \item In Chapter 2, we learn that if $\sum_{n=1}^\infty a_n$ converges conditionally, then 
  for any $r \in \mathbb{R}$, there exists a rearrangement of $\sum_{n=1}^\infty a_n$ that converges to $r$.
  \item In Chapter 3, we learn that $\mathbb{R}$ and $\emptyset$ are both open and closed, 
  but they are the only subsets in $\mathbb{R}$ with this property.
  \item In chapter 7, we talk about the Lebesgue Integral as a generalization of the Riemann Integral,
  which allows us to integrate more functions. This is cool, because it uses an intuitive
  idea about partitioning the $x$ axis as partitions of measures, instead of relying on 
  a defined partition like the Riemann Integral does.
\end{itemize}

\section{Important Theorems}
\subsection{5 Characterizations of Completeness}
\begin{theorem}
  \textbf{\textup{(Axiom of Completeness)}}
  Every nonempty set of real numbers that is bounded above has a least upper bound. 
\end{theorem}

\begin{theorem}
  \textbf{\textup{(Nested Interval Property)}}
  For each $n\in \mathbb{N}$, assume we are given a closed interval $I_n = \pbra{a_n, b_n} = \{x \in \mathbb{R}:a_n \leq x \leq b_n\}$.
  Assume also that each $I_n$ contains $I_{n+1}$. Then the resulting nested sequence of closed intervals
  \begin{equation*}
    I_1 \supseteq I_2 \supseteq I_3 \cdots
  \end{equation*}
  has a nonempty intersection, that is $\bigcap_{n=1}^\infty I_n \neq \emptyset$.
\end{theorem}

\begin{theorem}
  \textbf{\textup{(Monotone Convergence)}}
  If a sequence is monotonic and bounded, then it converges.
\end{theorem}

\begin{theorem}
  \textbf{\textup{(Bolzano-Weierstrass)}}
  Every bounded sequence contains a convergent subsequence.
\end{theorem}

\begin{theorem}
  \textbf{\textup{(Cauchy Criterion)}}
  A sequence converges if and only if it is a Cauchy Sequence.
  
  A sequence $(a_n)$ is called a Cauchy sequence if, for every $\epsilon > 0$,
  $\exists N \in \mathbb{N}$ such that whenever $m, n \geq N$, it follows that $\abs{a_n - a_m} < \epsilon$.
\end{theorem}

\subsection{Sequence Convergence}
\begin{theorem}
  \textbf{\textup{(Convergence of a Sequence)}}
  A sequence $\pbrac{a_n}$ converges to $a \in \mathbb{R}$
  if for every $\epsilon > 0$, $\exists N \in \mathbb{N}$ such that 
  whenever $n \geq N$, it follows that $\abs{a_n - a} < \epsilon$.
\end{theorem}

\begin{theorem}
  \textbf{\textup{(Cauchy Criterion)}}
  A sequence converges if and only if it is a Cauchy sequence,
  which is defined as a sequence $(a_n)$ that for every $\epsilon > 0$,
  $\exists N \in \mathbb{N}$ such that whenever $m, n \geq N$,
  it follows that $\abs{a_n - a_m} < \epsilon$.
\end{theorem}

\subsection{Function Continuity}
\begin{theorem}
  \textbf{\textup{(Characterizations of Continuity)}}
  Let $f : A \to \mathbb{R}$, and let $c \in A$ be a limit point of $A$.
  The function $f$ is continuous at $c$ if and only if any one of the following 
  conditions is met:
  \begin{enumerate}[label=(\roman*)]
    \item For all $\epsilon > 0$, $\exists \delta > 0$ such that for $x \in A, \abs{x - c} < \delta$
    implies $\abs{f(x) - f(c)} < \epsilon$
    \item $\lim_{x \to c} f(x) = f(c)$
    \item For all $V_\epsilon(f(c))$, there exists a $V_\delta(c)$ with the property that 
    $x \in A, x \in V_\delta(c)$ implies $f(x) \in V_\epsilon(f(c))$
    \item If $(x_n) \to c$, with $x_n \in A$, then $f(x_n) \to f(c)$
  \end{enumerate}
\end{theorem}

\begin{theorem}
  \textbf{\textup{(Uniform Continuity)}}
  A function $f:A\to \mathbb{R}$ is uniformly continuous on $A$
  if for every $\epsilon > 0$, $\exists \delta > 0$
  such that $\abs{x - y} < \delta$ implies $\abs{f(x) - f(y)} < \epsilon$.
\end{theorem}

\subsection{Function Convergence}
\begin{definition}
  \textbf{\textup{(Uniform Convergence)}}
  Let $f_n$ be a sequence of functions defined on a set $A \subseteq \mathbb{R}$.
  Then, $(f_n)$ converges uniformly on $A$ to a limit function $f$
  defined on $A$ if, for every $\epsilon > 0$, $\exists N \in \mathbb{N}$
  such that $\abs{f_n(x) - f(x)} < \epsilon$ whenever $n \geq N$ and $x \in A$.
\end{definition}

\section{Identities}
\begin{identity}
  \textbf{\textup{(Triangle Inequality)}}
  The triangle inequality states that for $x, y \in \mathbb{R}$,
  \begin{equation}
    \abs{x} - \abs{y} \leq \abs{x + y} \leq \abs{x} + \abs{y}
  \end{equation}
\end{identity}

\begin{identity}
  \textbf{\textup{(Geometric Series)}}
  \begin{equation}
    \sum_{k=0}^m ar^k = \frac{a(1-r^{m+1})}{(1-r)}
    \label{eq:geometric_series}
  \end{equation}
  and converges to 
  \begin{equation*}
    \lim_{m\to\infty} \sum_{k=0}^m ar^k = \frac{a}{1-r}
  \end{equation*}
  iff $\abs{r} < 1$.
\end{identity}