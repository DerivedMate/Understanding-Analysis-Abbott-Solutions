\appendix
\chapter{Extras}
\section{Useful Tools}

Collection of useful tools and methods to solve problems.

\begin{tip}
  Template for a proof that $(x_n) \rightarrow x$:
  \begin{itemize}
    \item Let $\epsilon > 0$ be arbitrary 
    \item Demonstrate a choice for $N \in \mathbb{N}$. 
    This step usually requires the most work, almost all of
    which is done prior to actually writing the formal proof.
    \item Now, show that $N$ works.
    \item Assume $n \geq N$
    \item With $N$ well chosen,  you should be able to show $\abs{x_n - x} < \epsilon$.
  \end{itemize}
\end{tip}

\section{Cool Things}
\begin{itemize}
  \item In Chapter 2, we learn that addition in infinite sums is not commutative.
\end{itemize}

\section{Important Theorems}
\subsection{5 Characterizations of Completeness}
\begin{theorem}
  \textbf{\textup{(Axiom of Completeness)}}
  Every nonempty set of real numbers that is bounded above has a least upper bound. 
\end{theorem}

\begin{theorem}
  \textbf{\textup{(Nested Interval Property)}}
  For each $n\in \mathbb{N}$, assume we are given a closed interval $I_n = \pbra{a_n, b_n} = \{x \in \mathbb{R}:a_n \leq x \leq b_n\}$.
  Assume also that each $I_n$ contains $I_{n+1}$. Then the resulting nested sequence of closed intervals
  \begin{equation*}
    I_1 \supseteq I_2 \supseteq I_3 \cdots
  \end{equation*}
  has a nonempty intersection, that is $\bigcap_{n=1}^\infty I_n \neq \emptyset$.
\end{theorem}

\begin{theorem}
  \textbf{\textup{(Monotone Convergence)}}
  If a sequence is monotonic and bounded, then it converges.
\end{theorem}

\begin{theorem}
  \textbf{\textup{(Bolzano-Weierstrass)}}
  Every bounded sequence contains a convergent subsequence.
\end{theorem}

\begin{theorem}
  \textbf{\textup{(Cauchy Criterion)}}
  A sequence converges if and only if it is a Cauchy Sequence.
  
  A sequence $(a_n)$ is called a Cauchy sequence if, for every $\epsilon > 0$,
  $\exists N \in \mathbb{N}$ such that whenever $m, n \geq N$, it follows that $\abs{a_n - a_m} < \epsilon$.
\end{theorem}