%% 7.2 %%

\setcounter{section}{1}
\section{The Definition of the Riemann Integral}
\setcounter{exercise}{0}

The partition theorems may be a little confusing at first to understand.
The basic idea is that if we have a refinement of some partition $P$,
then we have all the original points of $P$, with additional points 
in between.

Now, if we consider the $L(f, P)$, we see that adding more points 
in some interval $[x_k, x_{k+1}]$ 
will only yield estimates $m_k$ that are larger than the $\inf(f(x), x \in [x_k, x_{k+1}])$,
since in this interval, any $f(x)$ will be larger than the $\inf$.
Therefore, we can conclude 
\begin{equation*}
  L(f, P) \leq L(f, \text{refinement})
\end{equation*}
On the other hand, if we look at $U(f, P)$, we get the opposite relation.
The refinement produces a $U$ estimate that is less than the original, 
since if you have some maximum over some interval, adding more points 
gives you chance to take estimates that are less than the maximum.
Therefore, we conclude 
\begin{equation*}
  U(f, P) \geq U(f, \text{refinement})
\end{equation*}

There are some typos in this section:
\begin{itemize}
  \item 
  In Lemma 7.2.4, the author meant to say 
  if $Q = P_1 \cup P_2$, then $P_1 \subseteq Q, P_2, \subseteq Q$.

  \item 
  In Definition 7.2.5, where the \textit{Lower Integral} of $f$
  should be defined as
  \begin{equation*}
    L(f) = \sup\pbrac{
      L(f, P) : P \in \mathcal{P}
    }
  \end{equation*}
\end{itemize}

\bx{
For any partition $P$, we have $U(f) \geq L(f, P)$
since over any interval, the estimate for $f$ will be 
$\sup f(x)$ over this interval, and for $L(f, P)$,
it must be the $\inf f(x)$ over an interval that contains 
part of this interval, which must be less than $\sup f(x)$.

Then, $U(f, P) \geq L(f)$ since we chose an arbitrary $P$.
}

\bx{
We are considering $f(x) = 2x + 1$ over $x \in [1, 3]$.

\begin{figure}[H]
  \centering
  \def\coordinateScale{1.5}
  \def\padding{1}
  \def\tickWidth{2}

  \subfloat[$P = \pbrac{1, 3/2, 2, 3}$]{
    \begin{tikzpicture}
      \draw
        (0, 0) -- (3*\coordinateScale + \padding, 0)
        (0, 0) -- (0, 4*\coordinateScale + \padding)
        (0, 0) node[below left] {0}
      ;

      % Curve
      \draw
        (0, \coordinateScale) -- (3*\coordinateScale, 4*\coordinateScale)
      ;

      % Ticks
      \draw
        (\coordinateScale, \tickWidth pt) --
        (\coordinateScale, -\tickWidth pt) node[below] {1}
        (2*\coordinateScale, \tickWidth pt) --
        (2*\coordinateScale, -\tickWidth pt) node[below] {2}
        (3*\coordinateScale, \tickWidth pt) --
        (3*\coordinateScale, -\tickWidth pt) node[below] {3}

        (\tickWidth pt,    \coordinateScale) --
        (-\tickWidth pt,   \coordinateScale) node[left] {1}
        (\tickWidth pt,  2*\coordinateScale) --
        (-\tickWidth pt, 2*\coordinateScale) node[left] {3}
        (\tickWidth pt,  3*\coordinateScale) --
        (-\tickWidth pt, 3*\coordinateScale) node[left] {5}
        (\tickWidth pt,  4*\coordinateScale) --
        (-\tickWidth pt, 4*\coordinateScale) node[left] {7}
      ;

      % Estimates
      \draw
        (\coordinateScale, 0) -- (\coordinateScale, 4*\coordinateScale)
        (1.5*\coordinateScale, 0) -- (1.5*\coordinateScale, 4*\coordinateScale)
        (2*\coordinateScale, 0) -- (2*\coordinateScale, 4*\coordinateScale)
        (3*\coordinateScale, 0) -- (3*\coordinateScale, 4*\coordinateScale)
      ;

      % L estimates
      \draw[dashed]
        (\coordinateScale, 2*\coordinateScale) --
        (1.5*\coordinateScale, 2*\coordinateScale)
        (1.5*\coordinateScale, 2.5*\coordinateScale) --
        (2*\coordinateScale, 2.5*\coordinateScale)
        (2*\coordinateScale, 3*\coordinateScale) --
        (3*\coordinateScale, 3*\coordinateScale)
      ;

      % U estimates
      \draw[line width=1pt]
        (\coordinateScale, 2.5*\coordinateScale) --
        (1.5*\coordinateScale, 2.5*\coordinateScale)
        (1.5*\coordinateScale, 3*\coordinateScale) --
        (2*\coordinateScale, 3*\coordinateScale)
        (2*\coordinateScale, 4*\coordinateScale) --
        (3*\coordinateScale, 4*\coordinateScale)
      ;
    \end{tikzpicture}
  }
  \qquad
  \subfloat[$P = \pbrac{1, 3/2, 2, 5/2, 3}$]{
    \begin{tikzpicture}
      \draw
        (0, 0) -- (3*\coordinateScale + \padding, 0)
        (0, 0) -- (0, 4*\coordinateScale + \padding)
        (0, 0) node[below left] {0}
      ;

      % Curve
      \draw[->]
        (0, \coordinateScale) -- (3*\coordinateScale, 4*\coordinateScale)
      ;

      % Ticks
      \draw
        (\coordinateScale, \tickWidth pt) --
        (\coordinateScale, -\tickWidth pt) node[below] {1}
        (2*\coordinateScale, \tickWidth pt) --
        (2*\coordinateScale, -\tickWidth pt) node[below] {2}
        (3*\coordinateScale, \tickWidth pt) --
        (3*\coordinateScale, -\tickWidth pt) node[below] {3}

        (\tickWidth pt,    \coordinateScale) --
        (-\tickWidth pt,   \coordinateScale) node[left] {1}
        (\tickWidth pt,  2*\coordinateScale) --
        (-\tickWidth pt, 2*\coordinateScale) node[left] {3}
        (\tickWidth pt,  3*\coordinateScale) --
        (-\tickWidth pt, 3*\coordinateScale) node[left] {5}
        (\tickWidth pt,  4*\coordinateScale) --
        (-\tickWidth pt, 4*\coordinateScale) node[left] {7}
      ;

      % Estimates
      \draw
        (\coordinateScale, 0) -- (\coordinateScale, 4*\coordinateScale)
        (1.5*\coordinateScale, 0) -- (1.5*\coordinateScale, 4*\coordinateScale)
        (2*\coordinateScale, 0) -- (2*\coordinateScale, 4*\coordinateScale)
        (2.5*\coordinateScale, 0) -- (2.5*\coordinateScale, 4*\coordinateScale)
        (3*\coordinateScale, 0) -- (3*\coordinateScale, 4*\coordinateScale)
      ;

      % L estimates
      \draw[dashed]
        (\coordinateScale, 2*\coordinateScale) --
        (1.5*\coordinateScale, 2*\coordinateScale)
        (1.5*\coordinateScale, 2.5*\coordinateScale) --
        (2*\coordinateScale, 2.5*\coordinateScale)
        (2*\coordinateScale, 3*\coordinateScale) --
        (2.5*\coordinateScale, 3*\coordinateScale)
        (2.5*\coordinateScale, 3.5*\coordinateScale) --
        (3*\coordinateScale, 3.5*\coordinateScale)
      ;

      % U estimates
      \draw[line width=1pt]
        (\coordinateScale, 2.5*\coordinateScale) --
        (1.5*\coordinateScale, 2.5*\coordinateScale)
        (1.5*\coordinateScale, 3*\coordinateScale) --
        (2*\coordinateScale, 3*\coordinateScale)
        (2*\coordinateScale, 3.5*\coordinateScale) --
        (2.5*\coordinateScale, 3.5*\coordinateScale)
        (2.5*\coordinateScale, 4*\coordinateScale) --
        (3*\coordinateScale, 4*\coordinateScale)
      ;
    \end{tikzpicture}
  }
  \caption{
    Riemann Sum estimates for $f(x) = 2x +1$.
    Dashed lines indicate $L$, thick lines indicate $U$.
  }
  \label{chap7:fig:fx_partition}
\end{figure}

\ea{
\item We can compute
\begin{align*}
  L(f, P) &= \frac{1}{2}\pa{3 + 4} + 1 \cdot 5 = 8.5\\
  U(f, P) &= \frac{1}{2}\pa{4 + 5} + 1 \cdot 7 = 11.5\\
  U(f, P) - L(f, P) &= 3
\end{align*}

\item With an extra point $5/2$ in the partition,
\begin{align*}
  L(f, P) &= \frac{1}{2}\pa{3 + 4 + 5 + 6} = 9\\
  U(f, P) &= \frac{1}{2}\pa{4 + 5 + 6 + 7} = 11\\
  U(f, P) - L(f, P) &= 2
\end{align*}
So $U-L$ decreases.

\item If we add midpoints between all the current intervals, then we wil have 
\begin{equation*}
  U(f, P') - L(f, P') = \frac{7-3}{4} = 1 < 2
\end{equation*}
}
}

\bx{
The constant function satisfies $U(f) = L(f)$, since the $\sup, \inf$ over 
any interval is the same.

\begin{equation*}
  \int_{a}^{b} f(x) \dd{x} = b - a
\end{equation*}
}

\bx{
\ea{
\item \AFSOC $U(f) \neq L(f)$. Then we must have $U(f) > L(f)$.
Let $\epsilon = U(f) - L(f)$. We can find $P_n$ such that 
\begin{equation*}
  \abs{U(f, P_n) - L(f, P_n)} < \epsilon,
\end{equation*}
which means either $U(f, P_n) < U(f)$ or $L(f, P_n) > L(f)$,
both of which are contradictions, so therefore we conclude $U(f) = L(f)$.

\item We can find 
\begin{align*}
  L(f, P_n) &= \frac{1}{n}\pa{
    0 + \frac{1}{n} + \frac{2}{n} + \cdots + \frac{n-1}{n}
  } 
  = \frac{n(n-1)}{2n^2} = \frac{n-1}{2n}\\
  U(f, P_n) &= \frac{1}{n}\pa{
    \frac{1}{n} + \frac{2}{n} + \cdots + \frac{n-1}{n} + \frac{n}{n}
  }
  = \frac{n(n+1)}{2n^2} = \frac{n+1}{2n}\\
\end{align*}

\item We take the difference 
\begin{equation*}
  U(f, P_n) - L(f, P_n) =
  \frac{n+1-(n-1)}{2n} = \frac{1}{n}
\end{equation*}
so the difference $\to 0$, and therefore $f(x) = x$ is integrable on $[0, 1]$.
}
}

\bx{
Because $f_n \to f$ uniformly, hoose $n_1$ such that 
\begin{equation*}
  \abs{f_{n_1}(x) - f(x)} < \frac{\epsilon}{3\cdot(b-a)}
\end{equation*}
and since each $f_n$ is integrable, choose $n_2$ such that 
\begin{equation*}
  \abs{
    U(f_{n_1}, P_{n_2}) - L(f_{n_1}, P_{n_2})
  } < \frac{\epsilon}{3}.
\end{equation*}
Now, choose $n = \max(n_1, n_2)$,

Notice that 
\begin{align*}
  \abs{
    U(f, P_{n}) - U(f_n, P_{n})
  } 
  &\leq \sum_{x_k} \abs{f(x_k) - f_n(x_k)} \Delta x_k\\
  &<\sum_{x_k} \frac{\epsilon}{3(b-a)} \Delta x_k\\
  &=\frac{\epsilon}{3(b-a)} \sum_{x_k} \Delta x_k\\
  &=\frac{\epsilon}{3(b-a)} (b-a)\\
  &=\frac{\epsilon}{3}
\end{align*}
The key observation here is that over $[x_k, x_{k+1}]$,
$\abs{\sup f(x) - \sup f_n(x)} \leq \abs{f_{n}(x) - f(x)}$
since every point of $f_n$ is close to $f$.

A similar results holds for
\begin{equation*}
  \abs{
    L(f, P_{n_2}) - L(f_n, P_{n_2})
  }  < \frac{\epsilon}{3}
\end{equation*}

So therefore, we can say 
\begin{align*}
  \abs{
    U(f, P_n) - L(f, P_n)
  } &\leq 
  \abs{
    U(f, P_n) - U(f_n, P_n) + U(f_n, P_n) - L(f_n, P_n) - \pa{
      L(f, P_n) -
      L(f_n, P_n)
    }
  }\\
  &\leq 
  \abs{
    U(f, P_n) - U(f_n, P_n)
  } + 
  \abs{
    U(f_n, P_n) - L(f_n, P_n)
  } + 
  \abs{
    L(f, P_n) -
    L(f_n, P_n)
  }\\
  &\leq \frac{\epsilon}{3} +
  \frac{\epsilon}{3} +
  \frac{\epsilon}{3} = \epsilon
\end{align*}
\label{chap7:f_n_integrable}
}

\bx{
The key observation here is that for any interval $\pbra{x_k, x_{k+1}}$,
the $\sup$ of this set is at $x_{k+1}$, since that is the largest
element of this interval, and $f$ is increasing.
Likewise, $\inf f(x) = f(x_k)$.

Take $P_n$ to be a partition with all the subintervals
equal length $\Delta x = \frac{b-a}{n}$. We can compute
\begin{align*}
  U(f) - L(f)
  &= \sum_{k=0}^{n-1} \pbra{f(x_{k+1}) - f(x_{k})}(x_{k+1} - x_{k})\\
  &= \sum_{k=0}^{n-1} \pbra{f(x_{k+1}) - f(x_{k})}\Delta x\\
  &= \Delta x \pbra{
    \pa{
      f(x_n) - f(x_{n-1}) 
    } + 
    \pa{
      f(x_{n-1}) - f(x_{n-2})
    } + \cdots + 
    \pa{
      f(x_1) - f(x_0)
    }
  }\tag{Telescoping series}\\
  &= \frac{b-a}{n} \pbra{
    f(b) - f(a)
  }
\end{align*}
Now, let $M$ be $\abs{f(b)-f(a)}$.
Then choose 
\begin{equation*}
  n > \frac{(b-a)\pbra{f(b)-f(a)}}{\epsilon}
\end{equation*}
to satisfy $U(f, P_n) - L(f, P_n) < \epsilon$.

The official solutions for this problem is done wrong...the 
author got $\Delta x (b-a)$ at the end, but I think he meant 
$\Delta x \pbra{f(b) - f(a)}$. Of course, the idea is the same, 
but it's just confusing to see an error like that.
}