%% 7.3 %%

\section{Integrating Functions with Discontinuities}
\setcounter{exercise}{0}

\bx{
\ea{
\item Every partition contains some $x \in [x_k, x_{k+1}]$,
where $x_k < x_{k=1}$.

This means every interval contains some element $y \neq 1$, which 
will mean $\exists f(y) = 1$ in every interval, and thus 
\begin{equation*}
  L(f, P) = 1
\end{equation*} 

\item Just make a little interval around $x = 1$, so for example 
\begin{equation*}
  P = \pbrac{
    0, 
    1 - \frac{1}{15},
    1
  }
\end{equation*}
Then the difference is the interval 
$
\pbra{
  1 - \frac{1}{15},
  2
}
$
where we have $\Delta x = \frac{1}{15}$
and thus $U(f, P) = 1 + \frac{1}{15}$.

\item Construct
\begin{equation*}
  P_\epsilon = \pbrac{
    0, 
    1 - \frac{\epsilon}{2},
    1
  }
\end{equation*}
Then 
\begin{equation*}
  U(f, P_\epsilon) = 1 + \frac{\epsilon}{2}
\end{equation*}
}
}

\bx{
Fix some enumeration of the rationals.
Let $S_n = \pbrac{r_1, r_2, \dots, r_n}$, 
or the first $n$ rationals on this list.

Define 
\begin{equation*}
  g_n(x) = \begin{cases}
    1, &x \in S_n\\
    0, &x \not\in\mathbb{Q}
  \end{cases}
\end{equation*}
Then $g_n\to g$ since for any $x \in \mathbb{R}$,
if $x \not\in\mathbb{Q}$, then $g_n(x) = 0 = g(x)$.
Otherwise, we know $x = r_k$ for some $k$, and we can just 
choose $n = k$, and then $g_n(x) = 1 = g(x)$.
}

\bx{
We can make an $\epsilon$-neighborhood around each discontinuity
and remove them from $[a, b]$.

The remaining $[a, b]\setminus O$ must be continuous for $f$
since we just removed all the discontinuities.

Now, $[a, b] \setminus O$ is a compact set, becauase it is still bounded,
and is the union of a finite number of closed sets.
Therefore, $f$ is uniformly continuous on $[a, b] \setminus O$.

Since $f$ is uniformly continuous, we have 
\begin{equation*}
  \abs{
    f(x) - f(y)
  } < \epsilon
\end{equation*}
whenever $\abs{x - y} < \delta$.

This means for any $\epsilon$ challenge, 
we can define a partition such that the subintervals are small
enough around the discontinuities, and also so that
$\abs{x-y} < \delta$ is true for the $\delta$ we need.

Then we can bound $U-L$ at the discontinuities and the area 
that is continuous however we'd like.
}

\bx{
\ea{
\item Let the new value of this point be $M$.
We can do the putting a small $\epsilon'/2$-neighborhood around 
the new value We can then show that 
\begin{equation}
  U(f) - U(f, P_\epsilon') \leq \abs{M}\epsilon' < \epsilon/3,
  L(f) - L(f, P_\epsilon') \leq \abs{M}\epsilon' < \epsilon/3
\end{equation}
for appropriate values of $\epsilon'$.
Then, we can bound 
\begin{equation*}
  \abs{
    U(f, P_\epsilon') - 
    L(f, P_\epsilon')
  } \leq 
  \abs{
    U(f, P_\epsilon') - 
    U(f)
  } + 
  \abs{
    + U(f)
    - L(f)
  } +
  \abs{
    L(f, P_\epsilon')
    - L(f)
  } < \epsilon
\end{equation*}

\item For a finite number of $f$ changed, we can choose an even smaller $\epsilon'$
around each change, so something like if $M$ bounds the changes, and there are $N$ changes,
then choose 
\begin{equation*}
  N \cdot (2 \epsilon') M < \epsilon/3
\end{equation*}

\item We can just use Dirichlet's function idea, and change 
the value of $f$ at every rational $x \in [a, b]$.
}
}

\bx{
We can see that any interval $[x, y], x< y$ containing $1/n$
means $\exists z \neq 1/n$ so therefore 
the minimum $f(x)$ value on any interval $=0$, so we conclude
\begin{equation*}
  L(f) = 0
\end{equation*}

Define $P$ to be the partition so that for $P_N$,
we have 
\begin{equation*}
  P_N = \pbrac{
    0, 
    \frac{1}{N},
    1
  } \cup 
  S_r
\end{equation*}
where $S_r$ contains the points to add little intervals of 
radius $r = \frac{1}{N(N+1)}$ around each $1/k \in [0, 1]$
for $k < N$.

Then we have 

\begin{equation*}
  U(f, P_N) - L(f, P_N) = 
  r + 2r(N-2) + \frac{1}{N}
  = \frac{1}{N(N+1)} + \frac{2(N-2)}{N(N+1)} + \frac{1}{N}
\end{equation*}
which $\to 0$ as $N \to \infty$.

We showed earlier that $L(f) = 0$, so
\begin{equation*}
  \int_0^1 f(x) \dd{x} = L(f) = 0
\end{equation*}
\label{chap7:one_over_n_integrable}
}

\bx{
\ea{
\item 
Let the bound of $f$ on $[a, b]$ be $M$.
Choose a content zero set for the discontinuities 
so that their length of intervals is $< \epsilon/(4M)$
Also let $P_n$ be some partition
so that the continuous part of $f$, which is integrable, satisfies
\begin{equation*}
  U(f^\text{cont}, P_n) - 
  L(f^\text{cont}, P_n) < \epsilon/2
\end{equation*} 
and define $P_N$ 
be the union of $P_n$ with points to give us the content zero intervals.

Then we can show 
\begin{align*}
  U(f, P_N)  - L(f, P_N) 
  &< \epsilon/2 + 2M \cdot \frac{\epsilon}{4M}\\
  &= \epsilon
\end{align*}

\item For any finite set of $N$ elements, 
\begin{equation*}
  S_N = \pbrac{
    a_1, a_2, \dots, a_N
  },
\end{equation*}
define 
\begin{equation*}
  O_i = \pa{
    a_i - \frac{\epsilon}{2N},
    a_i + \frac{\epsilon}{2N}
  }
\end{equation*}
Then 
\begin{equation*}
  \sum_{k=1}^N \abs{O_k} = 
  N \cdot 2 \cdot \frac{\epsilon}{2N} = \epsilon
\end{equation*}

\item We know for any $m < n$, that $C_m \supseteq C_n$,
since $C_n$ is $C_m$ but with more open intervals removed.
In addition, we know that $C_m$ consists of $2^m$ closed intervals 
each with length $\frac{1}{3^m}$, so the total length of $C_m$ is
$\pa{\frac{2}{3}}^m$.

To get a finite zero set, we need open intervals. We can do this 
by adding a small $\epsilon'$ radius open set around each 
endpoint of the $2^m$ sets. Since this means adding length of 
\begin{equation*}
  2\epsilon' \cdot 2 \cdot 2^m,
\end{equation*}
since each interval is $2\epsilon'$ wide, 
and there are 2 endpoints per interval.
Now, choose $m$ such that 
\begin{equation*}
  \pa{\frac{2}{3}}^m < \epsilon
\end{equation*}
and let $\delta = \epsilon - \pa{\frac{2}{3}}^m$,
and then we can choose 
\begin{equation*}
  2\epsilon' \cdot 2 \cdot 2^m < \delta \Rightarrow \epsilon' < \frac{\delta}{2^{m+2}}
\end{equation*}
So the length of these open intervals will be less than $\epsilon$,
and therefore we can conclude that $C$ is a content zero set.

\item $h(x)$ is discontinuous at the $x \in C$ points, but we showed 
this has content zero, so we can conclude $h(x)$ is integrable.

We can find $x \not \in C$ for any open interval of $C$, so 
therefore we can conclude 
\begin{equation*}
  \int_0^1 h(x) \dd{x} = 0
\end{equation*}
}
}