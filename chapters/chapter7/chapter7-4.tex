%% 7.4 %%

\section{Properties of the Integral}
\setcounter{exercise}{0}

\bx{
\ea{
\item 
If both $\sup, \inf$ are negative or both nonnegative (i.e. same sign),
then 
\begin{equation*}
  M - m = M'-m'
\end{equation*}
If $\sup$ is positive and $\inf$ is negative, then $M' = M$, but
$m' = \inf \abs{f(x)} > \inf f(x) = m$, so 
\begin{equation*}
  M' - m' < M - m
\end{equation*}
\label{chap7:bound_u_l}

\item Using our work from part 
\ref{chap7:bound_u_l},
we can see that we can choose a partition $P$ so that 
\begin{equation*}
  \epsilon > 
    U(f, P) - L(f, P)
  \geq 
    U(\abs{f}, P) - L(\abs{f}, P)
\end{equation*}

\item It is always the case that $U(\abs{f}) \geq U(f)$,
so we can conclude that 
\begin{align*}
  U(\abs{f}) &\geq U(f)\\
  \int_a^b \abs{f} &\geq \int_a^b f\\
  \Rightarrow \abs{\int_a^b \abs{f}} 
  &\geq \abs{\int_a^b f}\\
  \int_a^b \abs{f}
  &\geq \abs{\int_a^b f} \tag{$\abs{f} \geq 0$}
\end{align*}
}
}

\bx{
For $c \leq a \leq b$, we can use the property
\begin{align*}
  \int_c^b f &= 
  \int_c^a f + \int_a^b f\\
  \int_c^b f -
  \int_c^a f &= \int_a^b f\\
  \int_c^b f +
  \int_a^c f &= \int_a^b f
\end{align*}
}

\bx{
From Exercise
\ref{chap7:f_n_integrable},
we saw that $f$ was integrable if we had $f_n \to f$ uniformly,
and each $f_n$ was integrable.

We need to still show that 
\begin{equation*}
  \lim_{n\to \infty} \int_a^b f_n = \int_a^b f
\end{equation*}
We can do this by first using the uniform convergence of $f_n$,
which gives us 
\begin{align*}
  \abs{f_n - f} &< \epsilon\\
  \Rightarrow 
  f - \epsilon &< f_n < f + \epsilon\\
  \int_a^b \pa{f - \epsilon} 
  &< 
  \int_a^b f_n
  <
  \int_a^b \pa{f + \epsilon}\\
  \Rightarrow 
  \abs{
    \int_a^b f_n -
    \int_a^b f 
  } &< \epsilon(b-a)
\end{align*}
so we can make their difference arbitrarily small.
}

\bx{
\ea{
\item False. Use Dirichlet's function, except define -1 for irrationals,
\begin{equation*}
  g(x) = \begin{cases}
    1, &x \in \mathbb{Q}\\
    -1, &\text{otherwise}
  \end{cases}
\end{equation*}
Then we have $\abs{g(x)} = 1$, which is integrable,
but we have $U(g) = 1, L(g) = -1$, which is not integrable.

\item False, consider $f(x) = 1, x = 1/n, 0$ otherwise we saw in Exercise 
\ref{chap7:one_over_n_integrable}, where the integral is still 0.

\item True. By the continuity of $g$, we can find a $\delta$ neighborhood
such that $x \in V_\delta(x_0) \Rightarrow \abs{g(x) - g(x_0)} < g(x_0)$,
which means $g(x) > 0$ on this interval.
Choose a partition that is entirely contained in this interval, then we have 
$L(g, P) > 0$, since that interval will yield a positive contribution
and the others are $\geq 0$ since $g \geq $.
Therefore, 
\begin{equation*}
  \int_a^b g = L(g) \geq L(g, P) > 0
\end{equation*}

\item We know $L(f) = \int_a^b f > 0$, 
which means there must have been some closed interval $[c, d]$
where $\inf f(x) > 0$. So we can choose this $\delta \in (\inf f(x), 0)$.

}
}

\bx{
\ea{
\item At any point $x \in [y_k, y_{k+1}]$, we have
\begin{equation*}
  (f+g)(x) \leq \sup f(x) + \sup g(x)
  \Rightarrow
  \sup (f+g)(x) \leq \sup f(x) + \sup g(x)
\end{equation*}
so we can conclude
\begin{equation*}
  U(f+g, P) = \sum_{x \in [y_k, y_{k+1}]} \sup (f+g)(x)
  \leq \sum_{x \in [y_k, y_{k+1}]} \sup f(x) + g(x)
  = U(f, P) + U(g, P)
\end{equation*}

We have equality when the supremums are at the same $x$ in $f, g$.
So a trivial example is $f(x) = x, g(x) = x$.

The lower sums will abide by
\begin{equation*}
  L(f+g, P) 
  \geq
  L(f, P) + L(g, P)
\end{equation*}
\label{chap7:sum_f_g_bound}

\item First, from our results in part 
\ref{chap7:sum_f_g_bound}, we have
\begin{align*}
  U(f+g, P) &\leq U(f, P) + U(g, P)\\
  L(f+g, P) &\geq L(f, P) + L(g, P)
\end{align*}
We can show $f+g$ is integrable by showing that 
\begin{align*}
  U(f+g, P_m) - L(f+g, P_m)
  &\leq \pbra{
    U(f, P_m) - L(f, P_m)
  } + \pbra{
    U(g, P_m) - L(g, P_m)
  }\\
  &< 
  \frac{\epsilon}{2} +
  \frac{\epsilon}{2} = \epsilon
\end{align*}
with the appropriate $P_m$ that bounds $f, g$.

Now, we can also conclude 
\begin{align*}
  U(f+g) &\leq U(f) + U(g)\\
  L(f+g) &\geq L(f) + L(g)
\end{align*}
which helps us show that
\begin{align*}
  \int_a^b f + \int_a^b g 
  &= L(f) + L(g)\\
  &\leq L(f+g) \\
  &\leq U(f+g)\\
  &\leq U(f) + U(g)\\
  &= \int_a^b f + \int_a^b g 
\end{align*}
so we can conclude 
\begin{equation*}
  \int_a^b f+g = U(f+g) =
  \int_a^b f + \int_a^b g 
\end{equation*}
}
}

\bx{
\ea{
\item We can define 
\begin{equation*}
  f_n(x) = \begin{cases}
    n^2, &x \in (0, 1/n)\\
    0, &\text{otherwise}
  \end{cases}
\end{equation*}
We see $f_n \to 0$, since we can always find 
$1/n < x$, which means $f_n(x) = 0$ for that particular $n$.
Then $\int_0^1 f_n(x) = n$, so the limit does not exist.
\label{chap7:part_int_unbounded}

\item We can use the same example in part
\ref{chap7:part_int_unbounded}, 
since the integral gives the sequence $i_n = n$.

\item We can modify our example in part
\ref{chap7:part_int_unbounded}
so that instead of a jump from $0$ to $n^2$, we define a line 
from $(1/2n, n^2) \to (1/n, 0)$.

Then this function will be continuous, but have 
$\int_0^1 f = \frac{3}{4}n$.

\begin{figure}[H]
  \centering
  \begin{tikzpicture}
    \draw 
      (0, 0) -- (2, 0) 
      (0, 0) -- (0, 2) node[left] {$n^2$}
    ;

    \draw 
      (0, 1.5) -- (0.75, 1.5) -- (1.5, 0)
      node[below] {$\frac{1}{n}$}
    ;
  \end{tikzpicture}
  \caption{The modified continuous $f_n(x)$}
  \label{chap7:continuous_fn}
\end{figure}

\item No, since if $f_n$ is bounded, we can just bound the integral by $M$.
}
}

\bx{
Let the bound of $g, g_n$ be $M$.

We want to bound
\begin{align*}
  \abs{
    \int_0^1 g_n 
    - \int_0^1 g
  } 
  &=
  \abs{
    \int_0^1 (g_n  - g)
  }\tag{$g, g_n$ integrable}\\
  &\leq 
  \int_0^1 \abs{
    g_n - g
  }\\
  &\leq 
  \int_0^\delta \abs{
    g_n - g
  } + 
  \int_\delta^1 \abs{
    g_n - g
  }
\end{align*}
We know $\abs{g_n - g} \leq 2M$, from their bounded property.
\begin{equation*}
  \int_0^\delta \abs{
    g_n - g
  } + 
  \int_\delta^1 \abs{
    g_n - g
  }
  \leq 
  \int_0^\delta 2M 
  + 
  \int_\delta^1 \abs{
    g_n - g
  }
\end{equation*}
and we can choose $\delta$ first, and then an $n$ such that 
\begin{equation*}
  \delta < \frac{\epsilon}{2M}, 
  \abs{g_n - g} < \frac{\epsilon}{2(1-\delta)}
\end{equation*}
and we can conclude
\begin{align*}
  \int_0^\delta 2M 
  + 
  \int_\delta^1 \abs{
    g_n - g
  } 
  &<
  \int_0^\delta 2M 
  + 
  \int_\delta^1 \frac{\epsilon}{2(1-\delta)}\\
  &= \frac{\epsilon}{2M} \cdot 2M + (1-\delta)\frac{\epsilon}{2(1-\delta)}\\
  &= \epsilon
\end{align*}
}