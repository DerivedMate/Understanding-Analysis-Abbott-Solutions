%% 7.6 %%

\section{Lebesgue's Criterion for Riemann Integrability}
\setcounter{exercise}{0}

\bx{
\ea{
\item $L(t, P) = 0$ since every interval will have 
an irrational number, so the $\inf = 0$ in all intervals.

\item The set of points $\geq \epsilon/2$ are 
\begin{align*}
  x &= 0\\
  x &= \frac{1}{1}\\
  x &= \frac{1}{2}\\
  x &= \frac{1}{3}\\
  &\cdots\\
  x &= \frac{1}{\floor*{2/\epsilon}}
\end{align*}
So we have 
\begin{equation*}
  \floor*{
    \frac{2}{\epsilon}
  } + 1
\end{equation*}
numbers on this list.

\item Choose a partition that is
\begin{equation*}
  \pbrac{
    0, \frac{
      1
    }{\floor{2/\epsilon}}
  } \cup \pbrac{
    V_{
      \epsilon^2/9
    }(x)
  }
\end{equation*}
Then we have 
\begin{align*}
  U(t, P_\epsilon) 
  &= \frac{\epsilon}{2} \cdot 1
  + \pbra{
    \pa{\floor*{\frac{2}{\epsilon}} + 1}
    \cdot \frac{\epsilon^2}{9}
  }\\
  &= \frac{\epsilon}{2} + \frac{
    \epsilon^2
  }{
    3
  }\\
  &\leq \frac{\epsilon}{2} +
  \frac{\epsilon}{3} \tag{for $\epsilon < 1$}\\
  &< \epsilon
\end{align*}
Note that $\epsilon \geq 1$ is trivial, since $\sup U(t, P) = 1$,
so any partition will work for $\epsilon \geq 1$.
}
}

\bx{
We can compute that $L(g) = 0$, 
since every interval must contain some $x \not\in C$,
so the $\inf = 0$ for any interval.

We know $C_n$ has closed sets with length $(2/3)^n$,
so we can define 
\begin{equation*}
  P_n = \text{finite union of closed intervals that make up } C_n
  \cup \text{the rest}
\end{equation*}
The only worry is that we can still include the endpoints 
of the closed intervals in ``the rest''. So what we do 
is we make the finite union of closed intervals slightly wider,
so that the length is still small. We can add $L$ length 
extensions at the endpoints, where 
\begin{equation*}
  2 \cdot 2^m \cdot L < \epsilon/2 
\end{equation*}

Then if we choose $\pa{\frac{2}{3}}^m < \epsilon/2$,
and then choose $L$ as indicated above,
\begin{align*}
  U(g, P_m) 
  &= 2^m\pa{
    \frac{1}{3^m} +
    2 L
  } + 0\\
  &= \pa{
    \frac{2}{3}
  }^m + 2 \cdot 2^m \cdot L \\
  &<  
  \epsilon/2 + \epsilon/2 = \epsilon
\end{align*}
\label{chap7:ex:cantor_integral}
}

\bx{
For any countable set, we have 
some enumeration, call it 
\begin{equation*}
  S = \pbrac{
    a_1, a_2, a_3, \dots
  }
\end{equation*}
We can then construct the interval 
\begin{equation*}
  O_n = \pa{
    a_n - \frac{\epsilon}{2^{n+1}},
    a_n + \frac{\epsilon}{2^{n+1}}
  }
\end{equation*}
Then 
\begin{align*}
  \sum_{n=1}^\infty \abs{O_n} 
  &=
  \sum_{n=1}^\infty 2 \cdot \frac{\epsilon}{2^{n+1}}\\
  &=
  \epsilon \sum_{n=1}^\infty \frac{1}{2^{n}}\\
  &=
  \epsilon
\end{align*}
}

\bx{
We can do the same strategy we did in Exericse
\ref{chap7:ex:cantor_integral},
where we choose a $C_n$ with really small intervals,
and in order to include everything in open intervals,
we extend each closed interval by a small amount.
Then, 
\begin{align}
  \sum \abs{O_n}
  &=
  2^n\pa{
    \frac{1}{3^n} +
    2 L
  }\\
  &< \epsilon \tag{See Exercise 
    \ref{chap7:ex:cantor_integral}
  }
\end{align}
}

\bx{
If $A$ and $B$ are measure zero, then we can find 
open covers $O^1_n, O^2_n$ such that 
\begin{align*}
  \sum \abs{O^1_n} &< \frac{\epsilon}{2}\\
  \sum \abs{O^2_n} &< \frac{\epsilon}{2}
\end{align*}
So then 
$A \cup B \subseteq \bigcup_n O^1_n \cup O^2_n$,
and
\begin{equation*}
  \sum \abs{O^1_n \cup O^2_n} 
  \leq
  \sum \abs{O^1_n} +
  \sum \abs{O^2_n} 
  <
  \frac{\epsilon}{2} +
  \frac{\epsilon}{2}
  - \epsilon
\end{equation*}

For a countable union, I feel like we can assign 
\begin{equation*}
  \sum_n \abs{O^k_n} < \frac{
    \epsilon
  }{2^k}
\end{equation*}
for each set $A_k$.

Then if we sum these together we get $< \epsilon$.

The official solutions seems to need some reordering of a 
summation...I'm not sure why we need that, since we can 
say the open cover is
\begin{equation*}
  \mathcal{O} = \bigcup_k \bigcup_n O^k_n
\end{equation*}
So if we want to figure out 
\begin{equation*}
  \sum \abs{\mathcal{O}} = 
  \sum_k \sum_n \abs{O^k_n} =
  \sum_k \epsilon/2^k = \epsilon
\end{equation*}
I think the more rigorous proof also shows the other
sum also equals $< \epsilon$.
}

Apparently the next exercises were already done 
in Section 4.6, but I'm going to do them again
for review.

\bx{
Let $x \in D_{\alpha_2}$,
then $\forall \delta$,
\begin{equation*}
  \abs{
    f(y) - f(z)
  } \geq \alpha_2 > \alpha_1
\end{equation*}
So therefore $f$ is also not $\alpha_1$-continuous,
and therefore $x \in D_{\alpha_1}$.
}

\bx{
\ea{
\item If we know that $f$ is continuous at $x$,
then we can always find a $\delta$ such that 
for $y, z \in V_\delta(x)$, that 
\begin{equation}
  \abs{
    f(y) - f(z)
  } < \epsilon < \alpha
\end{equation}
as long as we choose $\epsilon < \alpha$.

If $f$ is not $\alpha$-continuous at $x$, then
$f$ is discontinuous at that point, which means 
\begin{equation*}
  D_\alpha \subseteq D
\end{equation*}

\item This follows directly from the definition of 
continuity, since if we take the negation of 
the definition of continuity, we get that 
$\exists \epsilon_0 > 0$ such that $\abs{f(y) - f(z)} \geq \epsilon_0$
for any $y, z \in V_\delta(x)$.

We can write 
\begin{equation*}
  D = \bigcup_{n=1}^\infty D_{1/n}
\end{equation*}
since for any discontinuity at $x$,
we can find this $\epsilon_0$, which because of 
$\exists n, 1/n < \epsilon_0$,
which shows that $x \in D_{1/n}$.
}
}

\bx{
Consider some limit point of $D_\alpha$.
$\exists (x_n) \to x, x_n \in D_\alpha$.

Now \AFSOC $f$ is $\alpha$-continuous at $x$.
Then we know $\exists \delta, y, z \in V_\delta(x)$
implies $\abs{f(y) - f(z)} < \alpha$.

We know since $(x_n) \to x$, we can find $x_k, x_j$
such that $\abs{x_k - x}, \abs{x_j - x} < \delta$.
In this case, we have 
\begin{equation*}
  \abs{f(x_k) - f(x_j)} < \alpha
\end{equation*}
But this is a contradiction, since $x_k, x_j \in D_\alpha$,
which means for any $\delta$, we should be able to show that 
$\abs{f(y)-f(z)} \geq \alpha$. But with our assumption,
no matter what $\delta$ we choose, we can always produce some 
$x_k$ close enough to an element of $D_\alpha$ such that 
$\abs{f(x_k) - f(x)}$ is $< \alpha$.

Therefore, we must conclude that $f$ is $\alpha$-discontinuous 
at $x$, and therefore $x \in D_\alpha$.

Since $D_\alpha$ contains all of its limit points,
it must be closed.
}

\bx{
\AFSOC $f$ is not uniformly $\alpha$-continuous on $K$,
then we know 
\begin{equation*}
  \exists x_n, y_n, \lim \abs{x_n - y_n} = 0
  \text{ while }
  \abs{
    f(x_n) - f(y_n)
  } \geq \alpha
\end{equation*}
Since $K$ is compact, we know $x_n, y_n$ both have convergent
subsequences $(x_{n_k}), (y_{n_k})$ 
with limit $L_x = \lim x_{n_k}, L_y = \lim y_{n_k}$.
By the Algebraic Limit Theorem, we have 
\begin{equation*}
  \lim y_{n_k} = \lim \pa{
    \pa{
      y_{n_k} - x_{n_k}
    }
    + x_{n_k}
  }
  = 0 + L_x
\end{equation*}
Therefore, we have $L_y = L_x$, so the two subsequences converge 
to the same limit.

Now, since $f$ is $\alpha$-continuous at every point $x$, 
there $\exists \delta$ such that $y, z \in V_\delta(x)$
implies $\abs{f(y) - f(z)} < \alpha$.
Since $(x_{n_k}), (y_{n_k}) \to L$, we can find elements in both sequences 
that are $\delta/2$ away from $L$, call them $x', y'$,
which means $\abs{x' - y'} < \delta$,
and therefore 
\begin{equation*}
  \abs{
    f(x') - f(y')
  } < \alpha
\end{equation*}

But this is a contradiction, since we assumed this quantity $\geq \alpha$.

Therefore, we conclude that no such sequence exists, and therefore 
$\abs{x - y} < \delta$ implies $\abs{f(x) - f(y)} < \alpha$.
\label{chap7:ex:alpha_continuous}
}

\bx{
\ea{
\item Since we are assuming none of $I_n = \emptyset$,
we can use the NIP to show that there must be an $x$ 
in an infinite intersection of nonempty closed sets.

\item Since $x \in K$, $\exists G_{\alpha_0}$ that contains $x$.
Let $S = (x - \epsilon, x + \epsilon) \subseteq G_{\alpha_0}$.
We know such an $\epsilon$ exists since $G_{\alpha_0}$
is open.

We also know $\lim_n \abs{I_n} = 0$,
and since $K$ is compact and therefore bounded,
this means we can choose $n$ such that
\begin{equation*}
  \abs{
    K \cap I_n
  } < \epsilon
\end{equation*}

Now, we must have 
\begin{equation*}
  K \cap I_n \subseteq (x - \epsilon, x + \epsilon) \subseteq G_{\alpha_0}
\end{equation*}
But this is a contradiction since we assumed $K \cap I_n$
had no finite subcover.
}

\label{chap7:ex:finite_subcover}
}

\bx{
From the definition of measure zero, we know there exists 
some set of open intervals $G_k$ such that $D_\alpha$ is contained 
within it, and that 
\begin{equation*}
  \sum \abs{G_k} < \epsilon
\end{equation*} 

Now, since $D_\alpha$ is bounded by $[a, b]$,
and is closed, we know that $D_\alpha$ is compact, which means 
it has a finite subcover from $G_k$.

The last thing we have to do is make these open sets disjoint,
which we can do by merging any $G_k$ together 
that have any overlap, and we know their union is still an open set.

Finally, the measure of these open sets is still arbitrarily small, 
since first, we took a subset of the already measure zero set,
then we took unions of sets, which satisfy
\begin{equation*}
  \abs{G_1 \cup G_2} \leq \abs{G_1} + \abs{G_2}
\end{equation*}
}

\bx{
We know $f$ is $\alpha$-continuous on $K$, since $D_\alpha$
is removed. After removing a finite set of disjoint open intervals 
on $[a, b]$, we must have $K$ equal a union of a finite number of closed 
sets, which is still closed. Since $K$ is bounded, we conclude $K$ is compact.
We showed in Exercise
\ref{chap7:ex:alpha_continuous}
that if $f$ is $\alpha$-continuous over a compact set,
it must also be uniformly $\alpha$-continuous over it as well.
}

\bx{
We can choose the parition so that, for $\delta$ that makes $f$
uniformly $\alpha$-continuous on $[a, b]$,
\begin{equation*}
  P_\epsilon = \pbrac{
    \text{Even intervals of width }<\delta
  } \cup 
  \pa{
    \text{endpoints of }G_i
  }
\end{equation*}
Now, if we calculate 
\begin{align*}
  U(f, P_\epsilon) - L(f, P_\epsilon)
  &\leq \frac{\epsilon}{4M} \cdot 2M + (b-a) \frac{\epsilon}{2(b-a)}\\
  &\leq \frac{\epsilon}{2} +
  \frac{\epsilon}{2} = \epsilon
\end{align*}
For some explaining here, when we're inside the $G_i$ regions,
the maximum difference of $U-L$ is $2M$, since $M$ is the bound of $f$,
and we could potentially have $M - (-M) = 2M$.
The length of these intervals is conveniently $\epsilon/(4M)$
that we chose from earlier.

For the intervals outside of the $G_i$, they are at most width $\delta$,
so by $\alpha$-continuity, any $x, y$ in this interval will satisfy
\begin{equation*}
  \abs{
    f(x) - f(y)
  } < \alpha
\end{equation*}
As an upper bound, we can say $< \alpha$, and then 
say the sum of the length of these intervals is $(b-a)$.

Therefore, in this direction, we can conclude that $f$ is integrable.

\label{chap7:ex:p_epsilon_intervals}
}

\bx{
\ea{
\item 
If we have $U(f, P_\epsilon) - L(f, P_\epsilon) < \alpha\epsilon$,
then let $G_k$ be the intervals that contain $D_\alpha$.
Notice that because it is $\alpha$-discontinuous on these intervals,
any difference of $\abs{f(x) - f(y)} \geq \alpha$, so

If we now compute 
\begin{align*}
  \sum_k \alpha\abs{G_k} 
  &\leq \sum_k (M_k - m_k)\abs{G_k}\\
  &\leq \sum_j (M_j - m_j)\Delta x_j\\
  &= U(f, P_\epsilon) - L(f, P_\epsilon)\\
  &< \alpha\epsilon
\end{align*}
So we have 
\begin{equation*}
  \sum_k \alpha \abs{G_k} < \alpha\epsilon \Rightarrow
  \abs{D_\alpha} \leq \sum_k \abs{G_k} < \epsilon
\end{equation*}

\item Since we know each $D_\alpha$ as measure zero, we can define 
\begin{equation*}
  D = \bigcup_{n=1}^\infty D_{1/n}
\end{equation*}
Now, this has measure zero since every $\abs{D_{1/n}}$
can be bounded by $\epsilon/2^{n}$,
and thus $\abs{D} < \epsilon$.
}
}

\bx{
\ea{
\item 
\begin{align*}
  g'(0) &=
  \lim_{x \to 0} 
  \frac{
    g(x) - 0
  }{
    x - 0
  }\\
  &= 
  \lim_{x \to 0} 
  \frac{
    x^2 \sin (1/x)
  }{
    x
  }\\
  &=
  \lim_{x \to 0} 
  x\sin(1/x)\\
  &= 0
\end{align*}

\item We can compute 
\begin{equation*}
  g'(x) = 2x\sin(1/x) - \cos(1/x)
\end{equation*}

\item 
For any $\frac{1}{2\pi n}, \frac{1}{\pi + 2\pi n} \in (-\delta, \delta)$,
\begin{align*}
  g'\pa{
    \frac{1}{2\pi n}
  } &= 
  -1\\
  g'\pa{
    \frac{1}{\pi + 2\pi n}
  } &= 
  1
\end{align*}
Since $g'(x)$ is continuous between 
\begin{equation*}
  x \in \pbra{\frac{1}{\pi + 2\pi n}, \frac{1}{2\pi n}},
\end{equation*}
we see that $g'(x)$ attains every value between $-1, 1$ over $V_\delta(0)$.

We can conclude $g'$ is not continuous at $x=0$, because no matter
how small of a $V_\delta(0)$, we choose, we can always fine 
\begin{equation*}
  \abs{
    g'(x) - g'(0)
  } = 1
\end{equation*}
}
}

\bx{
\ea{
\item Eventually $f_n(c) = 0$, since $x \in C_n$ for some $n$.
So $\lim_{n\to \infty} f_n(c) = 0$.

\item If $x \not\in C$, then it will be part of some fragment of 
$g(x)$, which is constant after it is created. This fragment is 
also continuous.
}
}

\bx{
\ea{
\item $f'(x)$ is the differentiable $g(x)$ fragment we showed 
earlier was differentiable.

\item We can imagine $\abs{f(x)}$ as a sine wave that 
oscillates under the parabola $x^2$ centered around some endpoint $\in C$.

Then, $(x-c)^2$ is a parabola centered one of these endpoints,
which means it is the upper envelope for one of these sine waves.
See Figure 
\ref{chap7:fig:sine_parabola}
for how this envelope bounds the sine-parabola.

Now, this property implies that 
\begin{equation*}
  \abs{f(c+\delta)} \leq (c+\delta-c)^2 = \delta^2
\end{equation*}
so we can find
\begin{equation*}
  f'(c) = \frac{
    f(c+\delta) - f(c)
  }{
    c+\delta - c
  } = \frac{
    \delta^2
  }{
    \delta
  } = \delta
\end{equation*}
which we can make arbitrarily small,
so $f'(c) = 0$.

\begin{figure}[H]
  \centering
  \def\domainSize{0.2}
  \begin{tikzpicture}
    \begin{axis}[
      axis y line = middle,
      axis x line = bottom,
      scaled y ticks=false,
      xticklabels={,,,$c$,},
      yticklabels={,,}
    ]

    \addplot[
      domain=-\domainSize:\domainSize,
      samples=100
    ]{
      x^2
    };

    \addplot[
        domain=-\domainSize:\domainSize,
        samples=1000
    ]{
      abs(
        x^2*sin(180/(pi*x))
      )
    };
      
    \end{axis}
  \end{tikzpicture}
  \caption{Showing how $\abs{f(x)} \leq (x-c)^2$}
  \label{chap7:fig:sine_parabola}
\end{figure}

\item We have $f'(c) = 0$ for every $c \in C$.
Now, in any $\delta$-neighborhood of $c$, we can always find 
some $c' \not\in C$, which because is on an open set of a 
fragment of $g$, $\exists c''$ where $f'(c'') \neq 0$.

This means for any $\delta$,
$\exists c'' \in V_\delta(c)$ such that 
\begin{equation*}
  \abs{
    f'(c'') - f'(c)
  } = \abs{f'(c'')} \neq 0
\end{equation*}
So therefore, $f'(x)$ is not continuous.
}
}

\bx{
By Lebesgue's Theorem, we know that $f'$ is bounded ($\leq 1$), and $f'$
is not continuous at every $x \in C$, and we know $C$ is a measure zero set,
so therefore $f$ is Riemann-integrable.
}

\bx{
Now, the sum of the lengths of $C$ are the complement of 
the sum of the segments which we remove, which is 
\begin{equation*}
  \frac{1}{9} + 2 \cdot \frac{1}{27} + \cdots =
  \frac{
    1/9
  }{
    1 - 2/3
  } = \frac{1}{3}
\end{equation*}
So we can conclude $\lim_{n \to \infty} \abs{C_n} = \frac{2}{3}$.

I also wanted to do this directly...but it is more difficult.
The sequence for $\abs{C_k}$ looks like
\begin{equation*}
  \abs{C_0} = 1, \abs{C_k} = \abs{C_{k-1}} - \frac{2^{k-1}}{3^{k+1}}.
\end{equation*}
Now, I couldn't find a smart way to figure out this convergence, other
than noticing that 
\begin{equation*}
  \abs{C_k} = \abs{C_0} - \pa{
    \frac{1}{9} + 
    \frac{2}{27} + \cdots
    +
    \frac{2^{k-1}}{3^{k+1}}
  },
\end{equation*}
which just leads to the result we found earlier.
}