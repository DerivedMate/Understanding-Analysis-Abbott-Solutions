%% 3.3 %%

\section{Compact Sets}
\setcounter{exercise}{0}

\bx{
Since we know $K$ is compact, it must also be closed and bounded.

Since $K$ is bounded, it must have a least upper and largest lower bound, by
the Axiom of Completeness, which means $\sup K$ and $\inf K$ must exist. 
Now, we can construct subsequences of $K$ that converge to $\sup K, \inf K$,
since by the property of $\sup K$, for example, we can always find an element
of $K$ that is some $\epsilon$ close to it. So for example we can construct a
subsequence where $a_n = k_n, k_n \in K$ such that $\abs{\sup K - k} <
\frac{1}{2^n}$. The same logic applies to $\inf K$, so they are limit points
of $K$. Since $K$ is closed, $\sup K, \inf K \in K$.

\label{chap3:compact_sup_inf}
}

\bx{
Suppose we have some $K \subseteq \mathbb{R}$ that is closed and bounded.
We want to show that it is compact.

We know that any sequence of $K$
must be contained in $K$, which is bounded, so therefore 
by the Bolzano-Weierstrauss Theorem, we know that this sequence 
must have a convergent subsequence.
Since $K$ is also closed, this limit must be in $K$ as well.

This shows that $K$ is compact.
}

\bx{
We want to show the Cantor set is compact.

We know the Cantor set is $\subseteq [0, 1]$, so it is bounded.

Then, we know the complement of the Cantor set is 
\begin{equation*}
  \pa{-\infty, 0} \cup \pa{1, \infty} \cup 
  \pbra{
    \pa{\frac{1}{3}, \frac{2}{3}} \cup 
    \pa{\frac{1}{9}, \frac{2}{9}} \cup 
    \pa{\frac{1}{9}, \frac{2}{9}} \cup 
    \cdots
  }
\end{equation*}
which is the union of an arbitrary number of open sets, which we know is open.
Therefore, the Cantor set is the complement of an open set, which is closed.

Therefore, since the Cantor set is bounded and closed, we conclude it is compact.
}

\bx{
We have that $K$ is compact and $F$ is closed.
Since $K$ is compact, it is also bounded and closed.

If we take $K \cap F$, we know that this must also be bounded, since 
$x \in K \cap F \Rightarrow x \in K$.

The intersection of two closed sets is also closed, so $K \cap F$ is closed.

Therefore, $K \cap F$ is bounded and closed, and thus is compact.
}

\bx{
\ea{
\item We can find a sequence in $\mathbb{Q}$ that converges to $\sqrt{2}$, 
but we know that $\sqrt{2} \not\in \mathbb{Q}$, so $\mathbb{Q}$ is not compact.
\label{chap3:q_not_compact}
\item Again, similar to part \ref{chap3:q_not_compact}, we can find a sequence that 
converges to $\sqrt{2}/2 \not\in [0, 1] \cap \mathbb{Q}$.

\item Take the sequence $a_n = n$. There is no limit, so this sequence does not have a subsequence that converges.
\item $\mathbb{R} \cap [0, 1] = [0, 1]$ is closed and bounded, so it is compact.
\item This sequence converges to 0, but does not contain 0, so it is not closed and thus not compact.
\item Every subsequence of this set converges to 1, which is in the set, so therefore this set is compact.
}
}

\bx{
\ea{
\item We will prove this by induction.
\begin{itemize}
  \item \textbf{Base Case:} $n=1$, $C_1 = \pbra{0, \frac{1}{3}} \cup \pbra{\frac{2}{3}, 1}$
  We know a combination of two elements in $\pbra{0, \frac{1}{3}}$ and $\pbra{\frac{2}{3}, 1}$ covers $\pbra{0, \frac{4}{3}}$.
  Then, combination of two elements in the latter set covers $\pbra{\frac{4}{3}, 2}$.
  Therefore, two elements $x, y \in C_1$ can add up to any element $\in \pbra{0, 2}$.

  \item \textbf{Inductive Hypothesis:} Suppose for $k \geq 1$, any two elements of $C_k$ 
  can add up to any element $\in \pbra{0, 2}$.

  \item \textbf{Inductive Step} We know $C_{k+1} = C_k/3 + \pbrac{\frac{2}{3} + C_k/3}$,
  in other words, we are now missing the middle thirds of both the head and tail sets.
  What we want to show is that for the head set (and the same argument holds for the tail set),
  that with the middle third removed, we can still cover all of the original set, which means since
  the original sets are still covered, we can still cover $\pbra{0, 2}$.

  By the IH, any two elements of $C_k/3$ will cover $\pbra{0, \frac{2}{3}}$, 
  and any two elements of $\pbrac{\frac{2}{3} + C_k/3}$ covers $\pbra{\frac{4}{3}, 2}$.
  Choosing two elements from one of each set covers $\pbra{\frac{2}{3}, \frac{4}{3}}$.
  These three intervals cover $\pbra{0, 2}$.
\end{itemize}

\item The reason $(x_n)$, $(y_n)$ may not converge is they can be picked out of 
sets, jumping across different subsets of $C$ infinitely many times.

However, by the Bolzano-Weierstrauss Theorem, since $(x_n)$ is contained entirely 
in the Cantor set, which is bounded, then $(x_n)$ is also bounded. Therefore, it must 
contain a convergent subsequence. The same applies for $(y_n)$, and we can take their limits
$l_x, l_y$ such that $l_x + l_y = s$.
}
}

\bx{
\ea{
\item True. The intersection will be bounded, since we can take any bound of a set, which will bound the intersection,
and the intersection of an arbitrary number of closed sets is still closed.
Therefore, this arbitrary intersection of compact sets is closed and bounded, which means it is also compact.
\item False. Let $A = (0, 1), K = [0, 1]$, then $A \cap K = (0, 1)$, which is not closed so it is not compact.
\item True. This is the Nested Interval Property.
\item True. A finite set always closed, since there are no limit points, and bounded.
\item False. Choose an unbounded countable set like $a_n = n$.
}
}

\bx{
\ea{
\item If they both have finite subcovers, then we can union those two finite subcovers to get a 
finite subcover for $K$, so we need at least one of them to not have a finite subcover.
\item Create $I_{n+1}$ by bisecting $I_n$, and taking a half that has no finite subcover.
Such a half has to exist, because if both have finite subcovers, then the whole must have a finite subcover.
The interval will half in size every iteration.
\item By the Nested Interval Property, $\exists x\in K \forall x \in I_n$.
\item Since the interval sizes get arbitrarily small, we can find some $n_0$ such that $I_{n_0}$ fits entirely in $O_{\lambda_0}$.
We have reached a contradiction, because we can finitely cover $I_{n_0} \cap K$ by using $O_{\lambda_0}$.
}
}

\bx{
\ea{
\item Open cover where you take some $\epsilon > 0$ around all rational numbers,
and then union them together. No finite subcover exists, since that would bound $\mathbb{Q}$, which is not bounded.
\item We can construct an open cover like 
\begin{equation*}
  \pa{
    \bigcup_{i=1}^\infty (-0.5, \frac{\sqrt{2}}{2} - \frac{1}{i})
  }
  \cup
  \pa{
    \bigcup_{i=1}^\infty (\frac{\sqrt{2}}{2} + \frac{1}{i}, 1.1)
  }
\end{equation*}
This will contain all the rational numbers between $[0, 1]$, but needs an infinite 
number of sets to cover the entire set, because if we stop prematurely, we won't 
be able to capture the rationals that are very close to $\frac{\sqrt{2}}{2}$.
\setcounter{enumi}{4}
\item An open cover for this set is 
\begin{equation*}
  \bigcup_{i=1}^\infty \pa{1.1, 1-\frac{i-1}{i}}
\end{equation*}
We need all of these open sets, because otherwise if for some $N$ we stop, then 
we won't have the elements $< \frac{1}{N}$.
}
}

%% TODO: Not super sure about this solution...
\bx{
For any closed set with an interval $\pbra{a, b}$, we can make intervals $I_n = \pbra{a, b-1/n}$, and then 
union with $\pbra{b, b+1}$ to cover the set; with open intervals, we don't need the end.
However, we will need all of the intervals, otherwise 
we won't include all the elements close to $b$. Therefore, any \textit{clompact}
subset must be a finite set of isolated points.
}