%% 3.2 %%

\setcounter{section}{1}
\section{Open and Closed Sets}

\bx{
\ea{
\item We need a finite number of sets when we are choosing the minimum $\epsilon$
for our $V_\epsilon$. If we had an infinite number of sets, this minimum may not exist.

\item Let 
\begin{equation*}
  O_n = \pa{
    \sum_{i=1}^n \frac{1}{2^i},
    3 - \sum_{i=1}^n \frac{1}{2^i}
  }
\end{equation*} 
Then $\bigcap_{n=1}^\infty O_n = \pbra{1, 2}$.
}
}

\bx{
\ea{
\item 1 and $-1$ are the only limit points of $B$. For any fixed element of $B$, 
the distance between it and its neighbors is $\geq \frac{n}{n+1} - \frac{n+2}{n+3} = \frac{n}{(n+1)(n+3)}$,
so we can just choose $\epsilon$ smaller than this, and show that any element of $B$ is isolated.

For 1, we can show that for any $\epsilon > 0$, we can choose $\frac{1}{N+1} < \epsilon \Rightarrow N > \frac{1}{\epsilon} - 1$,
and we know $\frac{n}{n+1}$ for $n \geq N$ for even $n$ is in the $\epsilon$-neighborhood of 1.
Doing a similar analysis for negative terms and $-1$ yields the same result.

\item $B$ does not contain its limit points, so it is not closed.
\item $B$ is not an open set, continuous $\epsilon$ neighborhoods are not subsets of $B$.
\item All of $B$'s elements are isolated
\item $\overline{B} = B \cup \pbrac{-1, 1}$.
}
}

\bx{
\ea{
\item $\mathbb{Q}$ is not open, because it doesn't have irrationals that can be in the $\epsilon$-neighborhoods.
It is not closed, because it contains irrational limit points. Therefore it is \textbf{neither}.
\item $\mathbb{N}$ does not have any limit points, so it is \textbf{closed}.
\item $\mathbb{R}^+$ cannot be closed, because 0 is a limit point and not contained. 
It is \textbf{open} because every element has an $\epsilon$-neighborhood that is a subset.
\item Not closed, doesn't contain 0, a limit point. Not open, since $1$ has no $\epsilon$-neighborhood subset.
Therefore, \textbf{neither}.
\item The sequence converges, but this limit point is not in the set. No $\epsilon$-neighborhoods exist for certain $\epsilon$,
for certain elements, so not open. \textbf{Neither}.
}
}

\bx{
FOr any $\epsilon > 0$, we know $\exists N : n \geq N$ such that 
\begin{equation*}
  \abs{
    a_n - x
  } < \epsilon,
\end{equation*}
so every $\epsilon$-neighborhood of $x$ has points other than itself.
}

\bx{
If there exists exists such an $\epsilon$-neighborhood, then by the definition of a limit point,
since there are no other elements other than $x$ itself, then this is not a limit point, so it is isolated.
}

\bx{
If a set $F \subseteq \mathbb{R}$ is closed, then it contains all its limit points.
For any Cauchy Sequence in $F$, it is also convergent to some $L$, which we know is a limit 
point and thus must be in $F$.

If every Cauchy Sequence of an $F$ has its limit as an element of $F$,
then every limit point, which comes from the limit of some subsequence, which we know is a Cauchy Sequence.
From our original assumption, this limit must be in $F$, so $F$ contains all its limit points and is closed.
}

\bx{
\AFSOC an infinite number of $(x_n)$ terms not in $O$. 
Then since $(x_n) \rightarrow x$, $\epsilon = \text{distance of } x \text{ from $O$ boundary}$,
then $\forall N, : n \geq N$
we have that $\exists x_n : \abs{x_n - x} \geq \epsilon$, since we can choose some $x_n$ not in $O$.
This means this sequence does not converge. This contradicts our original assumption.
}

\bx{
\ea{
\item We want to show that $L$, which contains all the limit points of of $A$, is closed.
We can do this by showing all limit points of $L$ are in $L$.

Suppose we have some limit point $\ell$ of $L$, then this means some subsequence of $L$,
\begin{equation*}
  (l_n) \rightarrow \ell
\end{equation*}
By the definition of convergence, 
for any $\epsilon > 0$, we can find $N : n \geq N$ such that 
\begin{equation*}
  \abs{
    l_n - \ell
  } < \frac{\epsilon}{2}
\end{equation*}
Now, since $l_n$ are limit points of $A$, we know $\exists a \in A$ such that 
$a$ is arbitrarily close to $l_n$. Define a subsequence in $A$
\begin{equation*}
  \pbrac{
    a_n \in A, \abs{a_n - l_n} < \frac{\epsilon}{2}
  }
\end{equation*}
Then for $n \geq N$, 
\begin{equation*}
  \abs{
    a_n - \ell
  } < \abs{
    l_n - \ell
  } + \frac{\epsilon}{2}
  < 2 \cdot \frac{\epsilon}{2} = \epsilon,
\end{equation*}
which means $(a_n)$ converges to this $\ell$ as well, so $\ell$ is a limit point of $A$
and $\ell \in L$.

Therefore, we conclude $L$ contains all of its limit points, and therefore it is closed.

\label{chap3:part_L_closed}

\item For any limit point $\ell$ of $A \cup L$, it must the limit of some convergent subsequence of 
$A \cup L$. This subsequence will contain elements from $A$ and $L$.
What we can do is for every element in $L$, use a similar technique we did in part (\ref{chap3:part_L_closed}) to 
replace all the $x \in L$ subsequence elements with elements in $A$ instead, that are arbitrarily close enough.
Then, we have constructed a subsequence that entirely lies in $A$, so this limit point must be of $A$.
Therefore, all limit points of $A \cup L$ are limit points of $A$.

We can then conclude that $\overline{A} = A \cup L$ is a closed set, since all of its limit points are of $A$,
and those limit points are contained in $L$, which means $\overline{A}$ contains all of its limit points and 
is closed.
}
}

\bx{
\ea{
\item Suppose $y$ is a limit point of $A \cup B$, then there must exist a subsequence 
$(x_n), x_n \in A \cup B$ where $(x_n) \rightarrow y$. Now, this subsequence must contain either
an infinite number of elements from $A$ or $B$ (or both). 

WLOG, $(x_n)$ contains an infinite number of elements from $A$, then we know
$\exists \text{ subsequence } (x'_n) \rightarrow y$ where $x'_n \in A$. 
This means $y$ is a limit point of $A$.

Therefore, we conclude if $y$ is a limit point of $A \cup B$, $y$ is either a limit point of $A$ or $B$.

\item Let $L_S$ be the set of limit points for a set $S$.
\begin{align}
  \overline{A \cup B} 
  &= A \cup B \cup L_{A \cup B}\\
  &= A \cup B \cup \pa{L_A \cup L_B}
  &= (A \cup L_A) \cup (B \cup L_B)\\
  &= \overline{A} \cup \overline{B}
\end{align}

\label{chap3:distr_closure}

\item We notice that in our proof, we were able to find a subsequence that was entirely in one set.
Therefore, if it is possible to construct a subsequence that doesn't fit entirely in one set, 
then we can find a limit point that is not necessarily a limit point of an individual set.

With an infinite number of sets, we can take advantage of this property.

Suppose we have some $(a_n) \rightarrow L$, where $\forall n \, a_n \neq L$.
Then construct sets the following way,
\begin{equation*}
  S_n = \pbrac{a_n}
\end{equation*}
Now, $S_n$ has no limit points, since it only has a single point which is isolated.
Therefore, $\bigcup_{n=1}^\infty \overline{S_n} = \bigcup_{n=1}^\infty S_n$,
but we have $\overline{\bigcup_{n=1}^\infty S_n} = \pa{\bigcup_{n=1}^\infty S_n} \cup \pbrac{L}$,
and $L \not\in \bigcup_{n=1}^\infty S_n$. So the property in part (\ref{chap3:distr_closure})
does not apply for infinite sets.
}
}

\bx{
\ea{
\item A direct proof (double containment is another way to do it)
\begin{align*}
  x \in \pa{\bigcup_{\lambda \in \Lambda} E_\lambda}^c 
  &\Leftrightarrow \forall \lambda, x \not\in E_\lambda\\
  &\Leftrightarrow \forall \lambda, x \in E_\lambda^c\\
  &\Leftrightarrow x \in \bigcap_{\lambda \in \Lambda} E_\lambda^c
\end{align*}

\begin{align*}
  x \in \pa{\bigcap_{\lambda \in \Lambda} E_\lambda}^c 
  &\Leftrightarrow \exists \lambda, x \not\in E_\lambda\\
  &\Leftrightarrow \exists \lambda, x \in E_\lambda^c\\
  &\Leftrightarrow x \in \bigcup_{\lambda \in \Lambda} E_\lambda^c
\end{align*}

\item We want to show that 
\begin{enumerate}[label=(\roman*)]
  \item \textit{The union of a finite collection of closed sets is closed.}
  Suppose we have a collection of closed sets $\pbrac{E_\lambda, \lambda \in \Lambda}$,
  then, if we take the complement of the union of all these sets, by DeMorgan's, 
  we get the intersection of the complements of all these sets. The complements of all these sets 
  is open, and we know the intersection of a finite number of open sets is also open.
  Finally, taking the complement again, we must have a closed set,
  which is equal to our original union.

  \item \textit{The intersection of an arbitrary collection of closed sets is closed.}
  Take the intersection of these closed sets, and then take the complement.
  By DeMorgan's, we know have the union of the complement of these sets, which we know is open.
  We know that the union of an arbitrary number of open sets is also open.
  Finally, taking the complement of this entire expression again, we now have a closed set,
  which is equal to our original intersection.
\end{enumerate}
}
}

\bx{
If $s = \sup A$ exists, we have 2 cases. Either $s \in A \Rightarrow s \in \overline{A}$ since $A \subseteq \overline{A}$,
or, $s \not\in A$. In the second case, since we know for any $\epsilon > 0$, $\exists a \in A$ such that 
$a > s - \epsilon \Rightarrow \epsilon > \abs{s - a}$, we can construct a subsequence in $A$
that converges to $s$. This means $s$ is a limit point of $A$, and therefore $s \in \overline{A}$.
\label{chap3:ex_sup_closed}
}

\bx{
\ea{
\item True. $\overline{A}$ is closed, so $\overline{A}^c$ must be open. 
\item True. There is no $\epsilon$-neighborhood around this point that is contained in $A$.
\item False. Take the harmonic sequence $\pbrac{1/n}$.
\item True. See Exercise \ref{chap3:ex_sup_closed}
\item True. A finite set only contains isolated points, so therefore it has no limit points, and vacuously contains all of its limit points and is closed.
\item True. We know that around $q \in \mathbb{Q}$, exists some $\epsilon$-neighborhood around it 
that is contained in the set. Suppose we have an arbitrary $ r \in \mathbb{R}$, then we want to show 
it is contained in the $\epsilon$-neighborhood of some $q \in \mathbb{Q}$.
We can show this by contradiction. AFSOC $r$ is not in any of these $\epsilon$-neighborhoods.
That means $\forall \epsilon > 0, \abs{r - q} > \epsilon$, for all $q$.
But for any $\epsilon$, we can always find some rational number that is closer than $\epsilon$
to $r$, which means this statement is false. We have reached a contradiction, and 
must assume our original hypothesis was true.
}
}

\bx{
We can verify $\mathbb{R}$ is open because any $\epsilon$-neighborhood only contains 
elements of $\mathbb{R}$, so therefore $\subseteq \mathbb{R}$. In addition, any limit point $\in \mathbb{R}$,
so $\mathbb{R}$ also contains all of its limit points and is closed.

$\emptyset$ is closed and open by vacuity.

Now, we need to show that there are no other sets with this property.
We know the complement of an open set is closed and vice versa, so
\AFSOC $\exists A \neq \mathbb{R}, \emptyset$, then we know 
$\exists x \in A^c, \not\in A$.
Now, $x$ cannot be an isolated point, since then it would not have 
an $\epsilon$-neighborhood around it that is contained in $A^c$.
Therefore, we conclude $x$ must be in some continuous set $S$, where
it either 
\begin{enumerate}[label=(\roman*)]
  \item Has a $\sup S$. In this case, either $\sup S \in S$, in which case there does not 
  exist an $\epsilon$-neighborhood around $\sup S$, which means $S$ is not open,
  or $\sup S \not\in S$, and then $\sup S$ is a limit point, but then $S$ is not closed since 
  it doesn't contain all of its limit points. This case is not possible.
  \label{chap3:upper_bound_arg}
  \item Does not have an upper bound. Then look at the portion less than $x$
  and apply the argument in part (\ref{chap3:upper_bound_arg}) but with $\inf S$.
  It must have a lower bound, or else $A = \mathbb{R}$.
\end{enumerate}
Therefore, we reach a contradiction in all cases, and therefore we conclude that 
it is not possible for this $A$ to exist.
}

\bx{
\ea{
\item $\pbra{a, b} = \bigcap_{i=1}^n \pa{a - \frac{1}{n}, b + \frac{1}{n}}$.
Any $x \in \pbra{a, b}$ will be $< b + \frac{1}{n}$, and $> a - \frac{1}{n}$.
Now, let us consider some $y < a$. $y \not\in$ the set we created, bc suppose $\abs{y-a} = \epsilon$.
Then for $n' > \frac{1}{\epsilon}$, $a - \frac{1}{n'} > y$, so $y$ is not in this set.
The argument for an element larger than $b$ is symmetric. Therefore, we conclude the set we 
constructed is equivalent to $\pbrac{a, b}$, and is an intersection of a countable number of open sets.
\label{chap3:g_sigma_set}
\item We can write 
\begin{align}
  (a, b] &= \bigcup_{i=1}^n \pbra{a + \frac{1}{n}, b}\\
  (a, b] &= \bigcap_{i=1}^n \pa{a, b + \frac{1}{n}}
\end{align}
\item We know $\mathbb{Q}$ is countable, so just union all the sets containing only one element of $\mathbb{Q}$ together.
Since each set has one element which is an isolated point, each set is closed.
\begin{equation*}
  \bigcup_{q \in \mathbb{Q}} \pbrac{q}
\end{equation*}
We know that $\mathbb{Q}^c = \mathbb{I}$, and by DeMorgan's law, we know 
\begin{equation*}
  \pa{\bigcup_{i=1}^\infty S}^c = \bigcap_{i=1}^\infty S^c
\end{equation*}
Since $S$ are all closed, $S^c$ are all open. We can use the infinitely countable 
union of the construction of $\mathbb{Q}$ and then take the complement to get $\mathbb{I}$,
which by DeMorgan's is constructed as a countably infinite intersection of open sets.
}
\label{chap3:ex_f_g_sigma}
}