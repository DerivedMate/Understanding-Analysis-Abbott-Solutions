%% 3.4 %%

\section{Perfect Sets and Connected Sets}
\setcounter{exercise}{0}

\bx{
A perfect and a compact set are always closed, so their intersection is also closed.
Since a compact set is bounded, the intersection of it and another set must also be bounded.
Therefore, $P \cap K$ is a compact set. 

We cannot guarantee that $P \cap K$ does not have isolated points, so for example,
$\pbra{0, 1} \cap \pbrac{1/2} = \pbrac{1/2}$ which is not perfect.
}

\bx{
A perfect set cannot only consist of rationals, because it would be nonempty
and countable, since $\mathbb{Q}$ is countable, but this is impossible since
any nonempty perfect set is uncountable.
}

\bx{
\ea{
\item $x_1 \in C_1$ implies that $x_1$ is in a closed interval of size $1/3$, so 
we can choose any element in this interval such that $x \neq x_1$, and we must have 
$\abs{x - x_1} \leq 1/3$

\item We can make the argument for any $x_n \in C_n$, since $x_n$ must exist in a closed interval 
of size $\frac{1}{3}^n$, so we can any other element in this interval $x$, so that $\abs{x - x_n} \leq \frac{1}{3}^n$.
Now, we can construct a subsequence $(x_n)$ such that $(x_n) \rightarrow x$, and this shows that 
there are no isolated points in $C$. We know the Cantor set is closed from earlier exercises.
\label{chap3:cantor_no_isolated}
}
}

\bx{
\ea{
\item This set is bounded, and is also closed since it is an intersection of an arbitrary collection of closed sets.
Therefore, this construction is compact. 

This set is also perfect, because we can use 
the same argument from Exercise \ref{chap3:cantor_no_isolated} to show that there 
are no isolated points.

\item We can compute the 
\begin{itemize}
  \item \textbf{Length}: We will compute the removed interval lengths,
  \begin{equation*}
    \frac{1}{4} + 2 \cdot \frac{3}{32} + 4 \cdot \frac{27}{256} + \cdots
    = \frac{
      1/4
    }{
      1 - \frac{3}{2^2}
    } 
    = \boxed{1}
  \end{equation*}
  So this Cantor-like set has length 0.

  \item \textbf{Dimension}: We have $3 - 3 \cdot \frac{1}{4} = \frac{9}{4}$, so solving 
  \begin{equation*}
    3^x = \frac{9}{4} \Rightarrow \boxed{0.738}
  \end{equation*}
  This is ``larger'' in dimension than the ternary Cantor set.
\end{itemize}
}
}

\bx{
If we have that $A \subseteq U, B \subseteq V$ such that $U, V$ are disjoint open 
sets, then we know $A \cap B = \emptyset$. Therefore, if we want to show that 
they are separated, we just need to show that the limit points of $A$ are disjoint from $B$,
and vice versa.

\AFSOC that a limit point of $A, \ell_A \in B$. Then this $\ell_A$ is also a limit point of $U$, 
since $A \subseteq U$. Now, this $\ell_A \in V$, since $B \subseteq V$.
This means $\ell_A$ is $\epsilon$ far away from an element $\in U$, and since $\ell_A$
is also in the open set $V$, we must have that $\ell_A \in \pbra{v_1, v_2} \subseteq V$, where $v_1 < \ell_A < v_2$.
Let $\epsilon = \pa{\ell_A - v_1}/2$, then $\ell_A - \epsilon \in V$, and since $\ell_A$ is a limit point of $U$,
we must also have that $\ell_A - \epsilon \in U$. However, this is a contradiction, because we just 
showed that $\ell_A - \epsilon \in U$ and $\in V$, which means $U \cap V \neq \emptyset$, and they are 
not disjoint.
The same argument applies for the limit points of $B$ not being in $A$.
Therefore, we can conclude that $\overline{A} \cap B = \overline{B} \cap A = \emptyset$, 
and that $A, B$ are separated.
}

\bx{
($\Rightarrow$)
Suppose $E \subseteq \mathbb{R}$ is connected. Then consider some sets $A, B$
such that $A \cup B = E$, and $A, B$ are nonempty and disjoint.
\AFSOC every convergent sequence $(x_n)\rightarrow x$ with $(x_n)$ contained in $A \text{ or } B$,
$x \not\in \text{the other set}$. Then this must mean that every limit point of $A \text{ or } B$ is not in the other set, 
which means 
$\overline{A} \cap B = \overline{B} \cap A = \emptyset$,
and $A, B$ are separated. This means $E = A \cup B$ for separated $A, B$, which is a contradiction since 
we assumed $E$ was connected.

($\Leftarrow$)
Suppose for all nonempty and disjoint $A, B$ satisfying $E = A \cup B$, there always exists a convergent sequence $(x_n) \rightarrow x$
with $(x_n)$ contained in one of $A$, or $B$, and $x$ is an element of the other.
Suppose $(x_n)$ is contained within $A$. Then $x$ must be a limit point of $A$, since $x \in B$, and $A$ is disjoint of $B$ so $x \not\in A$.
This means $\overline{A} \cap B = x \cup S \neq \emptyset$, which means $A, B$ are not separated.
Therefore, we cannot find separated sets $A, B$ such that $E = A \cup B$, which means 
$E$ is not disconnected, and therefore is connected.
}

\bx{
\ea{ 
\item Take $E = \pa{-\infty, 1} \cup \pa{1, \infty}$, then the closure is $\mathbb{R}$, which is closed,
but this set is disconnected because $\pa{\pa{-\infty, 1} \cup \pbrac{1}} \cap \pa{1, \infty} = \emptyset$
and 
$\pa{-\infty, 1} \cap \pa{\pa{1, \infty} \cup \pbrac{1}} = \emptyset$
so these two sets are disconnected, and their union is equal to $E$.

\item If $A$ is connected, we can show that any limit points must already be in $A$, so $A = \overline{A}$ and $\overline{A}$ is still connected.
If $A$ is perfect, it is already closed, so it contains all of its limit points, and $\overline{A} = A$, and therefore $\overline{A}$ is still perfect.
}
}

\bx{
\ea{
\item Given any two rational $x, y$, WLOG $x < y$. Then $\exists r \in (x, y)$
such that $r \in \mathbb{I}$, i.e. it is not rational. Then we have $\mathbb{Q} = \pa{\mathbb{Q} \cap (-\infty, r)} \cup \pa{\mathbb{Q} \cap (r, \infty)}$,
and these two sets are disconnected.
\label{chap3:part_rational_disconnected}
\item Irrational numbers are also totally disconnected using the same argument in part 
(\ref{chap3:part_rational_disconnected})
}
}

\bx{
\ea{
\item We know in $C_n$, it consists of intervals of size $\frac{1}{3}^n$. If the intervals 
are smaller than $\epsilon$, i.e. $\frac{1}{3}^n < \epsilon$, then $x, y$ must be in different intervals.
\item If we know $x, y$ are in different intervals, then between their intervals there must exist removed intervals 
in the construction of $C$. We can take $z$ to be in one of these removed intervals.
Given any $(a, b), a < b$, we have a few cases. If $a$ or $b$ is not in $C$, then $(a, b)$ is not in $C$.
If $a, b \in C$, then we can use the argument we just made to find some $z \in (a, b)$, such that $z \not\in C$, which means
$(a, b) \not\subseteq C$.
\item For any $x, y \in C$, WLOG $x < y$, we can find $z \not\in C, x < z < y$ such that $C = \pa{C \cap (-\infty, z)} \cup \pa{C \cap (z, \infty)}$.
Therefore, $C$ is totally disconnected.
}
}

\bx{
\ea{
\item $O$ contains all the rational numbers (and some irrational numbers), so the complement $O^c$ must only consist of irrational numbers.
\item $F$ only consists of closed intervals. $F$ is totally disconnected, because for any $x, y \in \mathbb{I}, x < y$,
we can find a rational number $q$ where $q = r_n, \epsilon_n = 1/2^n < \abs{x-y}/2$, so it fits in between $x, y$.
Then we can make $F$ with the union of two open sets intersected with $F$ at that boundary.
\item We know $F$ is closed, since we are taking an arbitrary intersection of closed sets.
$F$ is not always perfect, since we can create isolated points, for example, by having 
open sets have end points that converge to some irrational number.
The issue with our construction was that we would allow the $\epsilon$ neighborhoods to get 
arbitrarily close to irrational numbers, and sort of ``squeeze'' them into isolated points.

One trivial way to prevent this is to have some sort of minimum neighborhood size, but then 
$F = \emptyset$.

A less trivial, but vague way of construction, is to just get rid of the isolated irrational points.
There can only be a countably infinite number of these, and there are uncountably many irrationals,
so in this case $F \neq \emptyset$.

\TODO Find a better construction for a perfect set of irrationals.
}
}